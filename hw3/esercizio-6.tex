\section*{Esercizio 6}

Sia $\mathcal{H} = L^2(\mathbb{R}, \mathbb{C})$. Si consideri l'operatore $T \in \mathcal{L}(\mathcal{H})$ definito da:
\[ (Tf)(x) = x \chi_{[0,1]}(x) f(x) \]
Indichiamo $h(x) = x \chi_{[0,1]}(x)$. $T$ è quindi l'operatore di moltiplicazione per la funzione $h$.

\paragraph{1. Norma di $T$:}
Per un operatore di moltiplicazione $M_h$ su $L^2$, la norma dell'operatore è data dalla norma infinito essenziale della funzione moltiplicatrice:
\[ \|T\| = \|h\|_{\infty} \]
Qui, $h(x)$ è definita da:
\[ h(x) = 
\begin{cases} 
x & \text{se } x \in [0, 1] \\
0 & \text{altrimenti}
\end{cases}
\]
Sull'intervallo $[0, 1]$, il valore massimo di $|x|$ è $1$. Al di fuori, la funzione è nulla. Pertanto:
\[ \|T\| = \sup_{x \in [0,1]} |x| = 1 \]

\paragraph{2. Spettro $\sigma(T)$:}

Lo spettro di un operatore di moltiplicazione per una funzione $h$ corrisponde all'immagine essenziale di tale funzione. 
L'immagine della funzione $h(x) = x \chi_{[0,1]}(x)$ è l'insieme dei valori raggiunti da $h$ su $\mathbb{R}$:
\begin{itemize}
    \item Per $x \in [0, 1]$, $h(x)$ percorre l'intervallo $[0, 1]$.
    \item Per $x \notin [0, 1]$, $h(x) = 0$, valore già incluso nell'intervallo precedente.
\end{itemize}
Lo spettro è quindi l'intervallo chiuso:
\[ \sigma(T) = [0, 1] \]

\paragraph{3. Spettro puntuale $\sigma_p(T)$:}
$\lambda$ è un autovalore di $T$ se esiste una funzione $f \in L^2(\mathbb{R})$ non nulla tale che $Tf = \lambda f$, ovvero:
\[ (h(x) - \lambda) f(x) = 0 \quad \text{quasi ovunque} \]
Questo impone che $f$ sia nulla quasi ovunque al di fuori dell'insieme $E_\lambda = \{x \in \mathbb{R} : h(x) = \lambda\}$. Affinché $f$ sia non nulla in $L^2$, la misura di Lebesgue di $E_\lambda$ deve essere strettamente positiva ($\mu(E_\lambda) > 0$).

\begin{itemize}
    \item \textbf{Caso $\lambda \in (0, 1]$:} L'equazione $h(x) = \lambda$ possiede un'unica soluzione $x = \lambda$. Un punto singolo ha misura nulla. Pertanto, questi valori non sono autovalori.
    \item \textbf{Caso $\lambda = 0$:} L'insieme $E_0 = \{x \in \mathbb{R} : h(x) = 0\}$ è uguale a $(-\infty, 0] \cup (1, +\infty)$. Questo insieme ha misura infinita (quindi $>0$). Possiamo definire una funzione $f = \chi_{[2,3]} \in L^2$ che soddisfa $Tf = 0$. 
\end{itemize}
L'unico autovalore dell'operatore è quindi:
\[ \sigma_p(T) = \{0\} \]

%%% Local Variables:
%%% mode: LaTeX
%%% TeX-engine: luatex
%%% ispell-local-dictionary: "italian"
%%% TeX-master: "main"
%%% End:

