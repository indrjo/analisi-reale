\section*{Esercizio 1}

Siano \(f, g \in \mathcal{S} \mathbb{R}^n\) e mostriamo che
\(f \ast g \in \mathcal{S}\mathbb{R}^n\). Consideriamo \(\alpha, \beta \in \mathbb{N}^n\).
\begin{align*}
  \left\lvert x^\alpha D^\beta(f \ast g)(x) \right\rvert
  &= \left\lvert x^\alpha (D^\beta f \ast g)(x) \right\rvert = \\
  &= \left\lvert x^\alpha \int_{\mathbb{R}^n}^{} D^\beta f(x - y) g(y) \mathrm d y
    \right\rvert \leq \\
  &\leq \int_{\mathbb{R}^n}^{} \left\lvert x^\alpha \right\rvert \left\lvert D^\beta f(x - y)
    \right\rvert \left\lvert g(y) \right\rvert \mathrm d y
\end{align*}
Poiché \(f, g \in \mathcal{S}\mathbb{R}^n\), allora comunque presi
\(\alpha_1, \alpha_2 \in \mathbb{N}^n\) esistono
\(C_f , C_g \in \mathbb{R}\) per cui per ogni \(z \in \mathbb{R}^n\) si ha
\begin{align*}
  & \left\lvert D^\beta f(z) \right\rvert \leq
    \frac{C_f}{\left\lvert z^{\alpha_1} \right\rvert} \\
  & \left\lvert g(z) \right\rvert \leq \frac{C_g}{\left\lvert z^{\alpha_2} \right\rvert} 
\end{align*}
Pertanto riprendendo dall'ultimo membro si ha
\begin{align*}
  \phantom{\left\lvert x^\alpha D^\beta(f \ast g)(x) \right\rvert}
  & \leq  \int_{\mathbb{R}^n}^{} \left\lvert x^\alpha \right\rvert \frac{C_f}{\left\lvert
    (x-y)^{\alpha_1} \right\rvert} \frac{C_g}{\left\lvert y^{\alpha_2}
    \right\rvert} \mathrm d y \leq \\
  & \leq C_f C_g \int_{\mathbb{R}^n}^{}  \frac{\left\lvert x^\alpha \right\rvert}{\left\lvert
    (x-y)^{\alpha_1} \right\rvert \left\lvert y^{\alpha_2}
    \right\rvert} \mathrm d y  
\end{align*}
Scegliendo ora \(\alpha_1\) e \(\alpha_2\) con la componenti minime
sufficientemente grande, abbiamo che l'ultimo integrale muore per
\(\left\lvert x \right\rvert \to +\infty\). Questo basta per dire che la
funzione liscia \(x \mapsto x^\alpha D^\beta f(x)\) è limitata.

%%% Local Variables:
%%% mode: LaTeX
%%% TeX-engine: luatex
%%% ispell-local-dictionary: "italian"
%%% TeX-master: "main"
%%% End:

