\section*{Esercizio 1}

Definito lo spazio di Schwartz $\mathcal{S}({\mbb{R}^n})$
\[ \mathcal{S}(\mathbb{R}^n) = \left\{ f \in \mathcal{C}^{\infty}(\mathbb{R}^n) \;:\; \forall\, \alpha, \beta \in \mathbb{N}_0^n,\ \sup_{x \in
      \mathbb{R}^n} \left| x^{\alpha} D^{\beta} f(x) \right| < +\infty \right\}.
\]
Dobbiamo dimostrare che sia chiuso per convoluzioni.

Chiaramente se $f,g\in\mathcal{S}(\mathbb{R}^n) $ allora sono due funzioni $\mathcal{C}^{\infty}$ e quindi la loro convoluzione è ancora $\mathcal{C}^{\infty}$. (Da spiegare?)\\

Fissiamo $\alpha,\beta\in\mbb{N}^n_0$ due multi-indici, avremo quindi:
\[
  \alpha=(\alpha_1,\dots,\alpha_n),\quad \alpha_i\in \mbb{N}_0
\]
\[
  \abs{\alpha}= \alpha_1+\alpha_2 +\dots +\alpha_n
\]
Allo stesso modo per $\beta$. Con $x^\alpha$ si intende:
\[
  x^\alpha\coloneq x_1^{\alpha_1}\cdot x_2^{\alpha_2} \cdot\dots\cdot x_n^{\alpha_n}\cdot
\]
mentre $D^{\beta}$ si intende:
\[
  D^\beta\coloneq\partial_{x_1}^{\beta_1} \cdots \partial_{x_n}^{\beta_n}.
\]
Ora dunque si tratta di mostrare che fissati qualsiasi $\alpha$ e
$\beta$ si ha
che:\[ \sup_{x \in \mathbb{R}^n} \left| x^{\alpha} D^{\beta} (f\ast g)(x) \right| < +\infty
\]

Poiché $f,g$ appartengono ad $\mathcal{S}(\mathbb{R}^n)$ le loro derivate sono maggiorabili
da qualsiasi polinomio, quindi posso invertire derivazione e
integrazione, avremo quindi che:
\[
  \begin{aligned}
    \bigl| x^{\alpha} D^{\beta} (f \ast g)(x) \bigr|
    &= \bigl| x^{\alpha} (D^{\beta} f \ast g)(x) \bigr| \\
    &= \Biggl| \int_{\mathbb{R}^n} x^{\alpha} (D^{\beta} f)(x - y) \, g(y) \, dy \Biggr| \\
    &\leq \int_{\mathbb{R}^n} \bigl| x^{\alpha} (D^{\beta} f)(x - y) \, g(y) \bigr| \, dy
  \end{aligned}
\]
Possiamo maggiorare $\abs{x^\alpha}$ sfruttando la norma euclidea
$\abs{x}$ dei vettori di $\mbb R^n$ infatti
avremo:\[ \abs{x^\alpha}=\abs{x_1^{\alpha_1}\cdot x_2^{\alpha_2} \cdot\dots\cdot x_n^{\alpha_n}}
\]
e, chiaramente: \[ \forall i,\quad \abs{x_i}\leq \abs{x}\leq 1+ \abs{x}
\]
quindi, essendo tutte potenze con esponente naturale
avremo:\[ \abs{x_i}^{\alpha_i}\leq (1+\abs{x})^{\alpha_i}
\]
quindi:
\[
  |x^{\alpha}| = \prod_{i=1}^{n} |x_i|^{\alpha_i} \le \prod_{i=1}^{n} (1 + |x|)^{\alpha_i} = (1 +
  |x|)^{\sum_{i=1}^{n} \alpha_i} = (1 + |x|)^{|\alpha|}.
\]
Utilizzando la disuguaglianza triangolare
\[
  |x| \le |x - y| + |y|
\]
e la proprietà
\[
  (1 + a + b)^N \le (1 + a)^N (1 + b)^N,
\]
ottieniamo la diseguaglianza:
\[
  (1 + |x|)^{|\alpha|} \le (1 + |x - y|)^{|\alpha|} (1 + |y|)^{|\alpha|}.
\]

Inseriamo queste maggiorazioni nell'espressione di interesse:
\[
  \begin{aligned}
    x^{\alpha} D^{\beta} (f \ast g)(x)
    &\le \int_{\mathbb{R}^n} (1 + |x|)^{|\alpha|} |(D^{\beta} f)(x - y)|\, |g(y)| \, dy \\
    &\le \int_{\mathbb{R}^n} 
      \underbrace{(1 + |x - y|)^{|\alpha|} |(D^{\beta} f)(x - y)|}_{A(x - y)} 
      \cdot 
      \underbrace{(1 + |y|)^{|\alpha|} |g(y)|}_{B(y)} 
      \, dy.
  \end{aligned}
\]
\begin{itemize}
\item \textbf{Per il termine $A$:} Poiché
  $f \in \mathcal{S}(\mathbb{R}^n)$, tutte le sue derivate decrescono più velocemente
  dell'inverso di qualsiasi polinomio.  Esiste quindi una costante $C_1$
  (indipendente da $x$ e $y$) tale che
  \[
    \sup_{z \in \mathbb{R}^n} (1 + |z|)^{|\alpha|} |D^{\beta} f(z)| \le C_1.
  \]

\item \textbf{Per il termine $B$:} Poiché
  $g \in \mathcal{S}(\mathbb{R}^n)$, anche la funzione $B(y)$ è a decadimento rapido.  Per
  garantire la convergenza dell'integrale, scegliamo un intero $M$ tale
  che $M > n$ (dimensione dello spazio).  Esiste quindi una costante
  $C_2$ tale che
  \[
    (1 + |y|)^{|\alpha| + M} |g(y)| \le C_2 \quad \Longrightarrow \quad (1 + |y|)^{|\alpha|} |g(y)| \le
    \frac{C_2}{(1 + |y|)^M}.
  \]
\end{itemize}

Sostituendo questi limiti nell'integrale, otteniamo
\[
  \begin{aligned}
    |x^{\alpha} D^{\beta} (f \ast g)(x)|
    &\le \int_{\mathbb{R}^n} A(x - y) \, B(y) \, dy \\
    &\le \int_{\mathbb{R}^n} C_1 \cdot \frac{C_2}{(1 + |y|)^M} \, dy \\
    &= C_1 C_2 \int_{\mathbb{R}^n} \frac{1}{(1 + |y|)^M} \, dy.
  \end{aligned}
\]

Poiché abbiamo scelto $M > n$, l'integrale
$\int_{\mathbb{R}^n} (1 + |y|)^{-M} \, dy$ converge. Pertanto, la quantità
$ |x^{\alpha} D^{\beta} (f \ast g)(x)| $ è limitata uniformemente in $x$.



%%% Local Variables:
%%% mode: LaTeX
%%% TeX-engine: luatex
%%% ispell-local-dictionary: "italian"
%%% TeX-master: "main"
%%% End:

