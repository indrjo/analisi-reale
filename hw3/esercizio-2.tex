\section*{Esercizio 2}

%Sia $\chi:\mathbb{R}\rightarrow\mathbb{R}$ la funzione caratteristica dell'intervallo $(0,1]$. Per
%ogni $n\in\mathbb{N}$, poniamo $\varphi_{n} := \ast^n \chi$.
Mostriamo per induzione su \(n \in \mathbb{N}\) che:
\[
  \varphi_{n}(x)=\frac{1}{(n-1)!}\sum_{k=0}^{n}\binom{n}{k}(-1)^{k}(x-k)_{+}^{n-1}
\]
%dove $(x)_{+}:=\max(0,x)$ e supponendo $0^{0}:=0$.
%Procediamo per induzione su $n \in \mathbb{N}$.

Per $n=1$, abbiamo:
\begin{align*}
  \varphi_{1}(x) &= \frac{1}{0!} \sum_{k=0}^{1} \binom{1}{k} (-1)^k (x-k)_+^0 = \\
  &= (x)_+^0 - (x-1)_+^0 = \mathbf{1}_{(0, +\infty)}(x) - \mathbf{1}_{(1,
    +\infty)}(x) = \chi(x) .
\end{align*}
% Utilizzando la convenzione $0^0=0$, $(x-k)_+^0$ vale $1$ se $x > k$ e $0$ altrimenti. Pertanto:
% \[
% \varphi_1(x) = \mathbf{1}_{x>0} - \mathbf{1}_{x>1} = \mathbf{1}_{0 < x \le 1} = \chi(x)
% \]
% La formula è quindi verificata per $n=1$.

Supponiamo la formula vera per un intero $n \ge 1$. Calcoliamo
$\varphi_{n+1} = \varphi_n * \chi$:
\[
  \varphi_{n+1}(x) = \int_{-\infty}^{+\infty} \varphi_n(t) \chi(x-t) \mathrm dt = \int_{x-1}^{x} \varphi_n(t)
  \mathrm dt
\]
Inserendo l'ipotesi induttiva:
\[
  \varphi_{n+1}(x) = \int_{x-1}^{x} \frac{1}{(n-1)!} \sum_{k=0}^{n} \binom{n}{k}
  (-1)^{k} (t-k)_+^{n-1} \, dt
\]
Per linearità dell'integrale e utilizzando la primitiva
$\int (t-k)_+^{n-1} dt = \frac{(t-k)_+^n}{n}$:
\[
  \varphi_{n+1}(x) = \frac{1}{n!} \sum_{k=0}^{n} \binom{n}{k} (-1)^{k} \left[
    (t-k)_+^n \right]_{x-1}^{x}
\]
\[
  \varphi_{n+1}(x) = \frac{1}{n!} \left( \sum_{k=0}^{n} \binom{n}{k} (-1)^{k}
    (x-k)_+^n - \sum_{k=0}^{n} \binom{n}{k} (-1)^{k} (x-1-k)_+^n \right)
\]
Nella seconda somma, effettuiamo il cambiamento di indice $j = k+1$:
\[
  \sum_{k=0}^{n} \binom{n}{k} (-1)^{k} (x-(k+1))_+^n = \sum_{j=1}^{n+1}
  \binom{n}{j-1} (-1)^{j-1} (x-j)_+^n
\]
Reintegrando questo nell'espressione (e sostituendo $j$ con $k$):
\[
  \varphi_{n+1}(x) = \frac{1}{n!} \left( \binom{n}{0}(x)_+^n + \sum_{k=1}^{n}
    \left[ \binom{n}{k} + \binom{n}{k-1} \right] (-1)^k (x-k)_+^n +
    \binom{n}{n}(-1)^{n+1}(x-(n+1))_+^n \right)
\]
In base all'identità di Pascal
$\binom{n}{k} + \binom{n}{k-1} = \binom{n+1}{k}$, e sapendo che
$\binom{n}{0} = \binom{n+1}{0} = 1$ e
$\binom{n}{n} = \binom{n+1}{n+1} = 1$, otteniamo:
\[
  \varphi_{n+1}(x) = \frac{1}{n!} \sum_{k=0}^{n+1} \binom{n+1}{k} (-1)^k
  (x-k)_+^n
\]
La proprietà è quindi dimostrata per induzione.


%%% Local Variables:
%%% mode: LaTeX
%%% TeX-engine: luatex
%%% ispell-local-dictionary: "italian"
%%% TeX-master: "main"
%%% End:

