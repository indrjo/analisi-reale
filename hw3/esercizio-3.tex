\section*{Esercizio 3}

Verifichiamo che
\(\left\langle \cdot, \cdot \right\rangle : \mathcal{H} \times \mathcal{H} \to K\) (dove
\(K\) è \(\mathbb{R}\) oppure \(\mathbb{C}\)) è un prodotto scalare.
\begin{enumerate}
\item Linearità nel primo argomento. Siano \(f, g, h \in \mathcal{H}\) e
  \(\lambda_1, \lambda_2 \in K\).
  \begin{align*}
    \left\langle \lambda_1 f + \lambda_2 g, h \right\rangle
    &= \int_{[0,1]} (\lambda_1 f + \lambda_2 g)'(x) h'(x) \mathrm d x = \\
    &= \lambda_1 \int_{[0,1]}f'(x) h'(x) \mathrm d x + \lambda_2 \int_{[0,1]}g'(x) h'(x)
      \mathrm d x = \\
    &=  \lambda_1 \left\langle f, h \right\rangle +  \lambda_2 \left\langle g, h \right\rangle .
  \end{align*}
\item Skew-simmetria. In realtà è proprio simmetrico, cioè
  \(\left\langle f, g \right\rangle = \left\langle g, f \right\rangle\) per ogni \(f, g \in \mathcal{H}\).
\item Sia \(f \in \mathcal{H}\). Allora
  \(\left\langle f, f \right\rangle = \left\lVert f' \right\rVert_2^2 \ge 0\). Così
  facendo, si deduce che se \(\left\langle f, f \right\rangle = 0\), allora,
  \(f' = 0\) quasi ovunque. Poiché \(f \in AC[0, 1]\), si ha che
  \[
    f(x) = f(0) + \int_{0}^{x} f'(t) \mathrm d t = 0 \qquad\text{per ogni } x \in
    [0, 1] .
  \]
\end{enumerate}

Rimane da provare che lo spazio vettoriale \(\mathcal{H}\) con la norma indotta
dal prodotto scalare
\[
  \left\lVert f \right\rVert := \sqrt{\left\langle f, f \right\rangle}
\]
è uno spazio di Banach. Sia quindi
\(\left\{ f_n \mid n \in \mathbb{N} \right\} \subseteq \mathcal{H}\) una successione di Cauchy rispetto
alla norma \(\left\lVert \cdot \right\rVert\). Se è così, allora
\(\left\{ f_n' \mid n \in \mathbb{N} \right\} \subseteq \mathcal{L}^2[0, 1]\) è una successione di
Cauchy rispetto alla norma \(\left\lVert \cdot \right\rVert_2\). Infatti
ricordare che
\[
  \left\lVert f \right\rVert = \left\lVert f' \right\rVert_2 .
\]
Poiché
\(\left(\mathcal{L}^2[0, 1], \left\lVert \cdot \right\rVert_2\right)\) è di Banach,
allora \(f_n'\) converge a un qualche elemento di
\(\mathcal{L}^2[0, 1]\) che indichiamo con \(f'\). A questo punto introduciamo
\begin{align*}
  & f : [0, 1] \to K \\
  & f(x) := \int_{0}^{x} f'(t) \mathrm d t .
\end{align*}
Questa funzione è assolutamente continua, \(f(0) = 0\) e la sua derivata
quasi ovunque è proprio \(f'\): pertanto \(f \in \mathcal{H}\). Concludiamo
\[
  \left\lVert f - f_n \right\rVert = \left\lVert (f - f_n)'
  \right\rVert_2 = \left\lVert f' - f_n' \right\rVert_2 \to 0 \quad \text{per
  } n \to +\infty .
\]

%%% Local Variables:
%%% mode: LaTeX
%%% TeX-engine: luatex
%%% ispell-local-dictionary: "italian"
%%% TeX-master: "main"
%%% End:

