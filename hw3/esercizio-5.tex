\section*{Esercizio 5}

Consideriamo il sottospazio
\[
  V := \left\{ p : [-1, 1] \to \mathbb{R} \mid p \text{ funzione polinomiale di grado
    } \leq N \right\} .
\]
È un sottospazio lineare chiuso (i sottospazi di dimensione finita di
uno spazio normato sono tutti chiusi.) Possiamo quindi usare il teorema
di proiezione: esiste uno e un solo \(p_f \in V\) tale che
\(\left\lVert f - p_f \right\rVert_2 = \inf_{p \in V} \left\lVert f - p
\right\rVert_2\).

Sia quindi \(f \in \mathcal{L}^2[-1, 1]\) e
\(p : [-1, 1] \to \mathbb{R}\) la funzione polinomiale
\[
  p(x) := \sum_{k=0}^{N} a_kx^k
\]
e calcoliamo la distanza da \(f\):
\begin{align*}
  \left\lVert f - p \right\rVert_2^2
  &= \int_{-1}^{1} (f(x) - p(x))^2 \mathrm d x = \\
  &= \int_{-1}^{1} p(x)^2 \mathrm d x + \int_{-1}^{1} f(x)^2 \mathrm d x - 2
    \sum_{k=0}^{N} a_k \int_{-1}^{1} x^k f(x) \mathrm d x 
\end{align*}
Troviamo i coefficienti \(a_0, \dots, a_N \in \mathbb{R}\) che minimizzano la
distanza sopra, i quali sono unicamente determinati come osservato
all'inizio. Derivando rispetto ad \(a_\mu\) e ponendo uguale a \(0\) si
trova il sistema
\[
  \sum_{j=0}^{N} a_j \int_{-1}^{1} x^{\mu + j} \mathrm d x = \int_{-1}^{1} x^\mu f(x)
  \mathrm d x \quad\text{per } \mu \in \left\{ 0, \dots, N \right\}
\]
nelle \(N\) incognite \(a_1, \dots, a_N\).

Osserviamo inoltre che in alcuni casi il sistema è più semplice da
risolvere. Se \(f : [-1, 1] \to \mathbb{R}\) è pari, allora la funzione polinomiale
\begin{align*}
  & q : [-1, 1] \to \mathbb{R} \\
  & q(x) := p_f(-x)
\end{align*}
soddisfa
\(\left\lVert f - q \right\rVert_2 = \left\lVert f - p_f
\right\rVert_2\). Questo però implica che \(p_f = q\) a causa
dell'unicità. Quindi \(p_f\) è pari.

Le funzioni polinomiali pari sono quelle i cui coefficienti dei monomi
di grado dispari sono nulli. Questo nel nostro caso significa rimangono
da determinare gli \(a_k\) con \(k\) pari.



%%% Local Variables:
%%% mode: LaTeX
%%% TeX-engine: luatex
%%% ispell-local-dictionary: "italian"
%%% TeX-master: "main"
%%% End:

