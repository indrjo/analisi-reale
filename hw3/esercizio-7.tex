\section*{Esercizio 7}

% \begin{enumerate}
% \item {\em Mostriamo che \(T \in \mathcal{L}(\mathcal{H})\) è autoaggiunto.} In questo caso,
%   ricordiamo che il prodotto scalare di \(\mathcal{H}\) è definito come
%   \[
%     \left\langle f, g \right\rangle = \int_{[0,1]}^{} f(x) \bar{g(x)} \mathrm d x .
%   \]
%   Verifichiamo quindi che
%   \(\left\langle Tf, g \right\rangle = \left\langle f, Tg \right\rangle\) per calcolo
%   diretto. Il primo membro è
%   \[
%     \left\langle Tf, g \right\rangle = \int_{[0, 1]}^{} \int_{[0, 1]}^{} K(x, y) f(y)
%     \bar{g(x)} \mathrm d y \mathrm d x
%   \]
%   mentre il secondo membro è
%   \[
%     \left\langle f, Tg \right\rangle = \int_{[0, 1]}^{} \int_{[0, 1]}^{} \bar{K(x, y)}
%     f(y) \bar{g(x)} \mathrm d y \mathrm d x
%   \]
%   Per concludere questa parte basta osservare che \(K = \bar{K}\).

% \item {\em \(T\) è compatto.}
%   \(K \in \mathcal{L}^2([0, 1] \times [0, 1])\) e quindi \(T\) è un operatore di
%   Hilbert-Schmidt, che è compatto. (Cfr. Proposizione 3.5.2 delle note
%   del corso.)

% \item {\em Trovare lo spettro di \(T\).} Osserviamo che essendo \(T\)
%   autoaggiunto, allora i suoi autovalori sono tutti reali
%   (cfr. Proposizione 3.6.6). Inoltre gli elementi non nulli dello spetto
%   sono autovalori di \(T\) (cfr. Proposizione 3.6.9). {\color{red} [Niente.]}
% \end{enumerate}

Sia $\mathcal{H} = L^2([0, 1], \mathbb{C})$. Si considerino $A, B \in L^2([0, 1], \mathbb{R})$ due funzioni linearmente indipendenti. Si definisce l'operatore $T \in \mathcal{L}(\mathcal{H})$ da:
\[ (Tf)(x) = \int_{0}^{1} K(x, y) f(y) \, dy \quad \text{con } K(x, y) = A(x)B(y) + A(y)B(x) \]
\\
Per un operatore integrale, il carattere autoaggiunto è garantito se il nucleo soddisfa
\\ $K(x, y) = \overline{K(y, x)}$. 
Qui, le funzioni $A$ et $B$ sono reali, quindi $K$ è reale. Infatti:
\[ K(y, x) = A(y)B(x) + A(x)B(y) = A(x)B(y) + A(y)B(x) = K(x, y) \]
Essendo il nucleo reale e simmetrico, l'operatore $T$ è autoaggiunto $T = T^*$.

\paragraph{1. Compattezza:}
Si ha:
\[ (Tf)(x) = A(x) \int_{0}^{1} B(y)f(y) \, dy + B(x) \int_{0}^{1} A(y)f(y) \, dy \]
Utilizzando il prodotto scalare standard $\langle f, g \rangle = \int f \bar{g}$, si ha:
\[ (Tf)(x) = \langle f, B \rangle A(x) + \langle f, A \rangle B(x) \]
Questo mostra che $\text{Im}(T) \subseteq \text{Vect}(A, B)$. L'immagine di $T$ è quindi di dimensione al massimo 2. 
$T$ è un operatore di rango finito, il che implica direttamente che è compatto.

\paragraph{2. Spettro $\sigma(T)$:}
Poiché $T$ è compatto in dimensione infinita, il suo spettro è composto dai suoi autovalori e da $0$. 
Sia $\lambda \neq 0$ un autovalore. Allora $f = \frac{1}{\lambda} Tf \in \text{Vect}(A, B)$. Poniamo $f = \alpha A + \beta B$. 
L'equazione $Tf = \lambda f$ porta al sistema lineare:
\[
\begin{cases}
\alpha \langle A, B \rangle + \beta \|B\|^2 = \lambda \alpha \\
\alpha \|A\|^2 + \beta \langle A, B \rangle = \lambda \beta
\end{cases}
\implies
\begin{pmatrix} \langle A, B \rangle - \lambda & \|B\|^2 \\ \|A\|^2 & \langle A, B \rangle - \lambda \end{pmatrix} \begin{pmatrix} \alpha \\ \beta \end{pmatrix} = 0
\]
Il determinante deve essere nullo: $(\langle A, B \rangle - \lambda)^2 - \|A\|^2 \|B\|^2 = 0$. 
Gli autovalori sono $\lambda_{\pm} = \langle A, B \rangle \pm \|A\| \|B\|$. 
Pertanto, $\sigma(T) = \{0, \langle A, B \rangle + \|A\| \|B\|, \langle A, B \rangle - \|A\| \|B\|\}$.

\paragraph{3. $\|K\|_{L^2} > \|T\|$:}
1. La norma dell'operatore è
$\|T\| = \max(|\lambda_+|, |\lambda_-|) = \|A\|\|B\| + |\langle A, B \rangle|$.
\\
2. La norma di Hilbert-Schmidt è
$\|K\|_{L^2}^2 = \iint |K(x,y)|^2 dx dy = 2\|A\|^2\|B\|^2 + 2\langle A, B \rangle^2$.
\\
Si nota che
$\|K\|_{L^2}^2 - \|T\|^2 = (\|A\|\|B\| - |\langle A, B \rangle|)^2$.  In base alla
diseguaglianza di Cauchy-Schwarz, poiché $A$ e $B$ sono linearmente
indipendenti, $|\langle A, B \rangle| < \|A\|\|B\|$.  La differenza è quindi
strettamente positiva: $\|K\|_{L^2} > \|T\|$. Le quantità non coincidono.



%%% Local Variables:
%%% mode: LaTeX
%%% TeX-engine: luatex
%%% ispell-local-dictionary: "italian"
%%% TeX-master: "main"
%%% End:

