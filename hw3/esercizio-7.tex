\section*{Esercizio 7}

\begin{enumerate}
\item {\em Mostriamo che \(T \in \mathcal{L}(\mathcal{H})\) è autoaggiunto.} In questo caso,
  ricordiamo che il prodotto scalare di \(\mathcal{H}\) è definito come
  \[
    \left\langle f, g \right\rangle = \int_{[0,1]}^{} f(x) \bar{g(x)} \mathrm d x .
  \]
  Verifichiamo quindi che
  \(\left\langle Tf, g \right\rangle = \left\langle f, Tg \right\rangle\) per calcolo
  diretto. Il primo membro è
  \[
    \left\langle Tf, g \right\rangle = \int_{[0, 1]}^{} \int_{[0, 1]}^{} K(x, y) f(y)
    \bar{g(x)} \mathrm d y \mathrm d x
  \]
  mentre il secondo membro è
  \[
    \left\langle f, Tg \right\rangle = \int_{[0, 1]}^{} \int_{[0, 1]}^{} \bar{K(x, y)}
    f(y) \bar{g(x)} \mathrm d y \mathrm d x
  \]
  Per concludere questa parte basta osservare che \(K = \bar{K}\).

\item {\em \(T\) è compatto.}
  \(K \in \mathcal{L}^2([0, 1] \times [0, 1])\) e quindi \(T\) è un operatore di
  Hilbert-Schmidt, che è compatto. (Cfr. Proposizione 3.5.2 delle note
  del corso.)

\item {\em Trovare lo spettro di \(T\).} Osserviamo che essendo \(T\)
  autoaggiunto, allora i suoi autovalori sono tutti reali
  (cfr. Proposizione 3.6.6). Inoltre gli elementi non nulli dello spetto
  sono autovalori di \(T\) (cfr. Proposizione 3.6.9). {\color{red} [Niente.]}
\end{enumerate}



%%% Local Variables:
%%% mode: LaTeX
%%% TeX-engine: luatex
%%% ispell-local-dictionary: "italian"
%%% TeX-master: "main"
%%% End:

