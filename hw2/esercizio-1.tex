\section*{Esercizio 1}

Al variare di \(p, p' > 0\), con \(p \ne p'\), troviamo un elemento di
\(L^p(\mathbb{R}^n)\) che non sta in \(L^{p'}(\mathbb{R}^n)\).

\begin{itemize}
\item {\em Caso \(p < p'\).} Consideriamo la funzione
  \(f : \mathbb{R}^2 \to \mathbb{R}\)
  \[
    f(x) :=
    \begin{cases}
      \frac{1}{\left\lVert x \right\rVert^{\alpha}}
      & \text{se } 0 < \left\lVert x \right\rVert \leq 1 \\
      0 & \text{altrimenti}
    \end{cases}
  \]
  e cerchiamo $\alpha$ in modo che $f \in L^p(\mathbb{R}^n)$ e
  $f \notin L^{p'}(\mathbb{R}^n)$. La norma $L^r$ della
  funzione $f$ è
  \[
    \left\lVert f \right\rVert_r^r = \int_{\mathbb{R}^n} \left\lvert f
    \right\rvert^r \mathrm dx = \int_{0 < \left\lVert x \right\rVert \leq 1} \frac{1}{\left\lvert x
      \right\rvert^{\alpha r}} \mathrm dx
  \]
  L'integrale converge se e solo se $\alpha r < n$. Quindi, se
  $\alpha \in \left[\frac{n}{p'}, \frac{n}{p}\right[$, si ha
  \(f \in L^p(\mathbb{R}^n)\) e \(f \notin L^{p'}(\mathbb{R}^n)\).

\item {\em Caso $p > p'$.} Consideriamo la funzione
  \(f : \mathbb{R}^n \to \mathbb{R}\)
  \[
    f(x) :=
    \begin{cases}
      \frac{1}{\left\lVert x \right\rVert^{\alpha}}
      & \text{se } \left\lVert x \right\rVert \geq 1 \\
      0 & \text{altrimenti}
    \end{cases}
  \]
  Come prima
  \[
    \|f\|_{r}^{r} = \int_{\mathbb{R}^n} \left\lvert f \right\rvert^r \mathrm dx =
    \int_{\left\lVert x \right\rVert \ge 1} \frac{1}{|x|^{\beta r}}
    \mathrm dx.
  \]
  L'integrale converge se e solo se \(\beta r > n\). Quindi se
  \(\frac{n}{p} < \beta \le \frac{n}{p'}\) si ha
  \(f \in L^{p}(\mathbb{R}^n)\) e \(f \notin L^{p'}(\mathbb{R}^n)\).

\item {\em Caso \(p = +\infty\).} La funzione \(f : \mathbb{R}^n \to \mathbb{R}\) costante a
  \(1\) è limitata ma non è un elemento di \(L^{p'}(\mathbb{R}^n)\).

\item {\em Caso \(p' = +\infty\).} Considera la funzione \(f : \mathbb{R}^n \to \mathbb{R}\)
  definita da 
  \[
    f(x) :=
    \begin{cases}
      \frac{1}{\left\lVert x \right\rVert^{\frac n p}}
      & \text{se } 0 < \left\lVert x \right\rVert \leq 1 \\
      0 & \text{altrimenti}
    \end{cases}
  \]
  Da considerazioni già fatte segue che \(f \in L^p(\mathbb{R}^n)\) e \(f \notin L^\infty(\mathbb{R}^n)\).
\end{itemize}

%%% Local Variables:
%%% mode: LaTeX
%%% TeX-engine: luatex
%%% ispell-local-dictionary: "italian"
%%% TeX-master: "main"
%%% End:

