\section*{Esercizio 2}

Troviamo \(f : [0, 1] \to \mathbb{R}\) che stia in tutti gli
\(L^p[0, 1]\), con \(p \ge 1\), ma non in \(L^\infty[0, 1]\). Consideriamo:
\[
  f(x) :=
  \begin{cases}
    0  & \text{se } x = 0 \\
    \ln\left(\frac{1}{x}\right) & \text{altrimenti}
  \end{cases}
\]
Qui $f \notin L^\infty([0,1])$, perché ad esempio \(\lim_{x \to 0^+} f(x) = +\infty\).
Inoltre
\[
  \left\lVert f \right\rVert_p^p = \int_0^1 \left(
    \ln\left(\frac{1}{x}\right) \right)^p \mathrm dx = \int_0^\infty t^p e^{-t}
  \mathrm dt < \infty.
\]
Quindi $f \in L^p([0,1])$ per ogni $p \ge 1$.



%%% Local Variables:
%%% mode: LaTeX
%%% TeX-engine: luatex
%%% ispell-local-dictionary: "italian"
%%% TeX-master: "main"
%%% End:

