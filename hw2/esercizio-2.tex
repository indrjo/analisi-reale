\section*{Esercizio 2}

\subsection*{Obiettivo}
Mostrare che $L^\infty([0,1]) \subsetneq \bigcap_{p \ge 1} L^p([0,1])$.

L'inclusione è banale poiché la misura è finita.

\subsection*{Richiamo: Funzione Gamma}
Per $z$ un numero complesso con $\text{Re}(z) > 0$, si definisce la
funzione Gamma:
\[
  \Gamma(z) = \int_0^{+\infty} t^{z-1} e^{-t} \, dt.
\]
Essa è assolutamente convergente per $\text{Re}(z) > 0$.

\subsection*{Dimostrazione della stretta inclusione}

Per la stretta inclusione, consideriamo la funzione:
\[
  f(x) = \ln\left(\frac{1}{x}\right).
\]

\begin{enumerate}[label=\arabic*., leftmargin=2em]
\item \textbf{$f \notin L^\infty([0,1])$:}
    
  Si ha
  \[
    \lim_{x \to 0^+} f(x) = \lim_{x \to 0^+} \ln\left(\frac{1}{x}\right) =
    +\infty.
  \]
  Ovvero, $\forall A > 0$, $\exists \delta > 0$ tale che $\forall x \in [0, 1]$:
  \[
    x < \delta \implies |f(x)| > A \implies \ln\left(\frac{1}{x}\right) > A.
  \]
  Quindi $f \notin L^\infty([0,1])$.
    
\item \textbf{$f \in L^p([0,1])$ per ogni $p \ge 1$:}
    
  Calcoliamo la norma $L^p$:
  \[
    \|f\|_p^p = \int_0^1 |f(x)|^p \, dx = \int_0^1 \left(
      \ln\left(\frac{1}{x}\right) \right)^p \, dx.
  \]
    
  Poniamo $t = \ln(1/x)$. Si ottiene:
  \[
    \|f\|_p^p = \int_0^\infty t^p e^{-t} \, dt = \Gamma(p+1) < \infty.
  \]
    
  Quindi $f \in L^p([0,1])$ per ogni $p \ge 1$.
\end{enumerate}

\subsection*{Conclusione}
Abbiamo trovato una funzione che appartiene a tutti gli $L^p$ ma non a
$L^\infty$, dimostrando che l'inclusione è stretta.

%%% Local Variables:
%%% mode: LaTeX
%%% TeX-engine: luatex
%%% ispell-local-dictionary: "italian"
%%% TeX-master: "main"
%%% End:

