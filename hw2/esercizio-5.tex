\section*{Esercizio 5}

\subsection*{Contesto}
Consideriamo $f \in L^1(\mathbb{R})$ e la successione delle traslazioni
$f_n(x) := f(x-n)$.

Sappiamo che per $p \in (1, +\infty)$, $f_n \rightharpoonup 0$ debolmente. Mostriamo qui che
questo è \textbf{falso} per $p=1$.

\subsection*{a)}

Sia $h \in C_c(\mathbb{R})$ una funzione continua a supporto compatto, diciamo
$\text{supp}(h) \subset [-T, T]$.

Studiamo il limite di $\displaystyle\int_{\mathbb{R}} f_n(x) h(x) \, dx$.

Con il cambio di variabile $y = x - n$:
\[
  \left| \int_{\mathbb{R}} f(x-n) h(x) \, dx \right| \le \|h\|_\infty \int_{-T}^T |f(x-n)| \, dx
  = \|h\|_\infty \int_{-T-n}^{T-n} |f(y)| \, dy.
\]

Poiché $f \in L^1(\mathbb{R})$:
\[
  \lim_{n \to \infty} \int_{-T-n}^{T-n} |f(y)| \, dy = 0.
\]

Quindi:
\[
  \lim_{n \to \infty} \int f_n h = 0 \quad \text{per ogni } h \in C_c(\mathbb{R}).
\]

\subsection*{b)}

Sia $g \in C_c(\mathbb{R})$ a supporto in $[-T, T]$. Sia $s(x) = \text{sgn}(g(x))$.

Definiamo:
\[
  h(x) = \sum_{k \in \mathbb{Z}} s(x - 2kT).
\]

\subsubsection*{1. Mostrare che $\|h\|_{\infty,\text{ess}} \le 1$}

I supporti dei termini $s(x-2kT)$ sono disgiunti, intervalli di
lunghezza $2T$.

Quindi, per ogni $x$, la somma contiene al massimo un termine non
nullo. Poiché $|s| \le \|s\|_\infty= 1$, si ha:
\[
  \|h\|_{\infty,\text{ess}} \le 1,
\]
quindi $h \in L^\infty(\mathbb{R})$.

\subsubsection*{2. Test sulla successione}

Consideriamo la successione traslata $g_n(x) = g(x - 2nT)$.
\begin{align*}
  \int_{\mathbb{R}} g_n(x) h(x) \, dx 
  &= \int_{\mathbb{R}} g(x - 2nT) \left( \sum_{k \in \mathbb{Z}} s(x - 2kT) \right) \, dx.
\end{align*}

L'unico termine della somma che si sovrappone al supporto di $g(x-2nT)$
è quello per $k=n$:
\begin{align*}
  &= \int_{\mathbb{R}} g(x - 2nT) s(x - 2nT) \, dx\\[0.5em]
  &= \int_{\mathbb{R}} g(y) s(y) \, dy\\[0.5em]
  &= \int_{\mathbb{R}} |g(y)| \, dy = \|g\|_1.
\end{align*}

\subsection*{c) Conclusione generale per densità}

Sia ora $f \in L^1(\mathbb{R})$ qualsiasi (non nulla).

Per densità delle funzioni continue a supporto compatto, per ogni
$\epsilon > 0$, esiste $g \in C_c(\mathbb{R})$ tale che:
\[
  \|f - g\|_1 < \epsilon.
\]

Costruiamo $h$ associata a questo $g$ come sopra.

Decomponiamo l'integrale:
\[
  \int f_n h = \int g_n h + \int (f_n - g_n) h.
\]

\begin{itemize}
\item Il primo termine vale $\|g\|_1$ (costante).
    
\item Il secondo termine è limitato per Hölder:
  \[
    \left| \int (f_n - g_n) h \right| \le \|f_n - g_n\|_1 \|h\|_\infty
    \le \|f - g\|_1 \cdot 1 < \epsilon.
  \]
\end{itemize}

Quindi, $\displaystyle\int f_n h$ rimane nell'intorno di $\|g\|_1 > 0$ e
non tende a 0.

\subsection*{Conclusione}
La successione $(f_n)$ non converge debolmente a 0 in $L^1(\mathbb{R})$.

%%% Local Variables:
%%% mode: LaTeX
%%% TeX-engine: luatex
%%% ispell-local-dictionary: "italian"
%%% TeX-master: "main"
%%% End:

