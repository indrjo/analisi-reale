\section*{Esercizio 9}
Siano $p\in[1,+\infty]$ e $q$ il suo esponente coniugato, cio\`e $\tfrac1p+\tfrac1q=1$.  
Siano $f\in L^{p}(\mathbb{R}^{n})$ e $g\in L^{q}(\mathbb{R}^{n})$.  
È noto che la convoluzione
\[
(f*g)(x)=\leb{\mathbb{R}^{n}}{y}{f(y)\,g(x-y)}
\]
\`e ben definita e uniformemente continua. Studiamo il comportamento della convoluzione per $x\rightarrow \infty$ per dimostrare che:
\[
(f\ast g)(x)\longrightarrow 0 \qquad \text{quando } \Abs{x}\to\infty.
\]
L'idea è sfruttare la densità delle funzioni $\mathcal{C}^\infty_c$, sia in $\mathcal{L}^p$ che in $\mathcal{L}^q$, così che quando $x\rightarrow \infty$ almeno uno degli argomenti dell'integrale esca dal supporto di $f$ o $g$.
\subsection*{Caso $1<p<\infty$}

Poich\'e $1<q<\infty$, lo spazio $\mathcal{C}^\infty_c(\mathbb{R}^n)$\`e denso in $\mathcal{L}^{q}(\mathbb{R}^{n})$.  
Dato $\varepsilon>0$, esiste dunque $\varphi\in \mathcal{C}^{\infty}_{c}(\mbb R^{n})$ tale che
\[
\Abs{g-\varphi}_q\le \frac{\varepsilon}{\Abs{f}_p}.
\]

Usando l'associatività della convoluzione, per ogni $x\in\mbb R^{n}$ vale la stima
\[
\abs{(f\ast g)(x)}
  \le \abs{(f\ast\varphi)(x)} + \abs{(f\ast (g-\varphi))(x)}.
\]

Il secondo termine si controlla con la disuguaglianza di H\"older:
\[
\abs{(f*(g-\varphi))(x)}
  \le \Abs{f}_p\Abs{g-\varphi}_q
  \le \varepsilon.
\]

Per il primo termine scriviamo, denotando con $K=$supp$(\varphi)$:
\[
(f*\varphi)(x)=\leb{K}{y}{\varphi(y)\,f(x-y)}.
\]
Applicando di nuovo H\"older:
\[
\abs{(f*\varphi)(x)}
 \le \Abs{\varphi}_q\,\Abs{f\cdot\chi_{x-K}}_p,
\]
dove $x-K=\set{x-y:y\in K}$.

Poich\'e $K$ è compatto, esiste $R>0$ tale che
\[
\Abs{f\,\chi_{B(0,R)^c}}_p<\delta
\]
per un $\delta>0$ arbitrario (essendo $f\in \mathcal{L}^{p}$).\\ 
Dunque definitivamente $\Abs{x}>R+$diam$(K)$ (supponendo $\Abs{x}\rightarrow\infty)$ si ha $x-K\subset B(0,R)^c$, dunque
\[
\Abs{f\,\chi_{x-K}}_p<\delta.
\]

Segue quindi
\[
\abs{(f\ast\varphi)(x)} \le \Abs{\varphi}_q\cdot\delta \xrightarrow[\Abs{x}\to\infty]{} 0.
\]

Combinando le due stime otteniamo
\[
\limsup_{\Abs{x}\to\infty} \abs{(f\ast g)(x)}
   \le \varepsilon.
\]
Poiché $\varepsilon>0$ è arbitrario, segue che
\[
(f\ast g)(x)\longrightarrow 0 \qquad \text{quando } \Abs{x}\to\infty.
\]

\subsection*{Caso $p=1$}

Il risultato non vale in generale.  
Un controesempio semplice è il seguente:
\[
f \in \mathcal{L}^{1}, \Abs{f}_1\neq 0,
\qquad g \equiv 1 \in \mathcal{L}^{\infty}.
\]
Allora per ogni $x$:
\[
(f*g)(x)=\leb{\mbb R^{n}}{y}{f(y)} = \Abs{f}_1,
\]
una costante non nulla.  
Dunque $(f\ast g)(x)\not\to 0$ all'infinito.

\subsection*{Caso $p=+\infty$}

Consideriamo l'esempio simmetrico:
\[
f\equiv 1 \in \mathcal{L}^{\infty}, \qquad g\in \mathcal{L}^{1},\Abs{g}_1\neq 0,
\]
Allora
\[
(f*g)(x)=\leb{\mbb R^{n}}{y}{g(y)}=\Abs{g}_1,
\]
ancora una costante non nulla.

Evidentemente il problema è che le funzioni in $\mathcal{L}^\infty$ non sono tutte approssimabili con funzioni a supporto compatto, che garantiscono invece la nullità in intorni dell'infinito. 


%%% Local Variables:
%%% mode: LaTeX
%%% TeX-engine: luatex
%%% ispell-local-dictionary: "italian"
%%% TeX-master: "main"
%%% End:
