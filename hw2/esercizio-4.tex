\section*{Esercizio 4}

{\em Mostriamo che $f \in L^p(E)$.} Esiste una sottosuccessione
\(\left\{ f_{n_k} \mid k \in \mathbb{N} \right\}\) convergente a \(f\)
quasi ovunque.  Poiché $x \mapsto |x|^p$ è continua, abbiamo in
particolare che
\[
  \lim_{k \to \infty} |f_{n_k}(x)|^p = |f(x)|^p.
\]
Impieghiamo il lemma di Fatou:
\[
  \int_E |f|^p \leq \liminf_{k \to \infty} \int_E |f_{n_k}|^p \le 1 .
\]

{\em Mostriamo la convergenza forte in $L^r$ ($1 < r < p$).} Sappiamo
che $f_n, f \in L^1(E) \cap L^p(E)$ e:
\[
  \|f_n - f\|_p \leq \|f_n\|_p + \|f\|_p \leq 2.
\]
Poniamo $g_n = f_n - f$.  Scegliamo $\mu, \lambda > 0$ tali che
$\mu + \lambda = 1$ e $r = \mu + \lambda p$. Quindi:
\[
  \|g_n\|_r^r = \int_E |g_n|^r = \int_E |g_n|^\mu \cdot
  |g_n|^{p\lambda}.
\]
Introduciamo ora $a := \frac{1}{\mu} > 1$ e
$b := \frac{1}{\lambda} > 1$. Si verifica immediatamente che
\(|g_n|^\mu \in L^a(E)\):
\[
  \int_E |g_n|^{\mu a} = \int_E |g_n|^{\mu \cdot \frac{1}{\mu}} = \int_E
  |g_n| = \|g_n\|_1 < +\infty
\]
e che \(|g_n|^{p\lambda} \in L^b(E)\):
\[
  \int_E |g_n|^{p\lambda b} = \int_E |g_n|^{p\lambda \cdot
    \frac{1}{\lambda}} = \int_E |g_n|^p = \|g_n\|_p^p < +\infty
\]
Applichiamo la disuguaglianza di Hölder ora:
\begin{align*}
  \|g_n\|_r^r 
  &\le \left( \int_E |g_n|^{\mu a} \right)^{1/a} \left( \int_E |g_n|^{p\lambda b} \right)^{1/b}\\[0.5em]
  &= \left( \int_E |g_n| \right)^\mu \cdot \left( \int_E |g_n|^p \right)^\lambda\\[0.5em]
  &= \|g_n\|_1^\mu \cdot \|g_n\|_p^{p\lambda}.
\end{align*}
All'ultimo membro abbiamo \(\left\lVert g_n \right\rVert_1 \to 0\) per
\(n \to +\infty\) e \(\left\lVert g_n \right\rVert_p \leq 2\). Possiamo
quindi concludere che \(f_n \to f\) in \(L^r(E)\).

{\em Proviamo la convergenza debole in $L^p$.} Scegliamo una coppia di
esponenti coniugati \(p,q \geq 1\) e facciamo vedere che per ogni
\(g \in L^q(E)\) si ha
\[
  \int_E (f_n - f) g \to 0 \text{ per } n \to +\infty .
\]
Consideriamo due casi.
\begin{enumerate}
\item $g \in L^1(E) \cap L^\infty(E)$.
  \[
    \left| \int_E (f_n - f)g \right| \le \int_E |f_n - f| |g| \le \|g\|_\infty \int_E |f_n -
    f| = \|g\|_\infty \|f_n - f\|_1 \xrightarrow[n \to \infty]{} 0.
  \]
  Quindi:
  \[
    \lim_{n \to \infty} \int_E (f_n - f)g = 0 \implies f_n \rightharpoonup f \text{ in } L^p(E)
    \text{ per } g \in L^1 \cap L^\infty.
  \]

\item Caso generale: $g \in L^q$. Per densità di
  $L^1(E) \cap L^\infty(E)$ in $L^q$, esiste $h \in L^1(E) \cap L^\infty(E)$ tale che:
  \[
    \|g - h\|_q \le \varepsilon \quad \text{per ogni } \varepsilon > 0.
  \]
  Decomponiamo:
  \begin{align*}
    \left| \int_E (f_n - f)g \right| 
    &\le \left| \int_E (f_n - f)(g - h) + (f_n - f)h \right|\\[0.5em]
    &\le \left| \int_E (f_n - f)(g - h) \right| + \left| \int_E (f_n - f)h \right|.
  \end{align*}
  Applicando Hölder al primo addendo dell'ultimo membro, possiamo
  proseguire:
  \[
    \le \|f_n - f\|_p \|g - h\|_q + \left| \int_E (f_n - f)h \right| \le 2 \varepsilon +
    \left| \int_E (f_n - f)h \right|,
  \]
  dove ricordiamo $\|f_n - f\|_p \le 2$. Quindi:
  \[
    \lim_{n \to \infty} \left| \int_E (f_n - f)g \right| \le 2 \varepsilon + \lim_{n \to \infty}
    \left| \int_E (f_n - f)h \right|.
  \]
  Il secondo termine tende a 0 (caso precedente). Pertanto:
  \[
    \lim_{n \to \infty} \left| \int_E (f_n - f)g \right| \le 2 \varepsilon, \quad \forall \varepsilon > 0.
  \]
  Facendo tendere $\varepsilon \to 0$, otteniamo:
  \[
    \lim_{n \to \infty} \int_E (f_n - f)g = 0 \implies f_n \rightharpoonup f \text{ in } L^p(E).
  \]
\end{enumerate}




%%% Local Variables:
%%% mode: LaTeX
%%% TeX-engine: luatex
%%% ispell-local-dictionary: "italian"
%%% TeX-master: "main"
%%% End:

