\section*{Esercizio 4}

\subsection*{Ipotesi}
Sia $\{f_n\}$ limitata in $L^p(E)$ ($p \in (1, +\infty)$) e convergente
fortemente a $f$ in $L^1(E)$. Esiste allora una sottosuccessione
convergente quasi ovunque.

\subsection*{Parte 1: $f \in L^p(E)$}

Si ha $\displaystyle\lim_{k \to \infty} f_{n_k}(x) = f(x)$ quasi ovunque, il
che implica:
\[
  \lim_{k \to \infty} |f_{n_k}(x)| = |f(x)|.
\]

Poiché $x \mapsto |x|^p$ è continua:
\[
  \lim_{k \to \infty} |f_{n_k}(x)|^p = |f(x)|^p.
\]

Dal lemma di Fatou:
\[
  \int_E |f|^p \leq \int_E \liminf_{k \to \infty} |f_{n_k}|^p \le \liminf_{k \to \infty}
  \int_E |f_{n_k}|^p \le 1.
\]

Quindi $f \in L^p(E)$.

\subsection*{Parte 2: Convergenza forte in $L^r$ ($1 < r < p$)}

Sappiamo che $f_n, f \in L^1(E) \cap L^p(E)$ e:
\[
  \|f_n - f\|_p \leq \|f_n\|_p + \|f\|_p \leq 2.
\]

Poniamo $g_n = f_n - f$.

Poiché $L^1(E)$ e $L^p(E)$ sono spazi vettoriali, si ha
$g_n \in L^1(E) \cap L^p(E)$.

Da $g_n \in L^1 \cap L^p$, esistono $\mu, \lambda > 0$ tali che
$\mu + \lambda = 1$ e $r = 1 \cdot \mu + p \cdot \lambda$.

Quindi:
\[
  \|g_n\|_r^r = \int_E |g_n|^r = \int_E |g_n|^\mu \cdot
  |g_n|^{p\lambda}.
\]

Poniamo $a = \dfrac{1}{\mu} > 1$ e $b = \dfrac{1}{\lambda} > 1$.

Verifichiamo:
\begin{itemize}[leftmargin=2em]
\item
  $\displaystyle\int_E |g_n|^{\mu a} = \int_E |g_n|^{\mu \cdot
    \frac{1}{\mu}} = \int_E |g_n| = \|g_n\|_1 < +\infty \implies
  |g_n|^\mu \in L^a(E)$.
    
\item
  $\displaystyle\int_E |g_n|^{p\lambda b} = \int_E |g_n|^{p\lambda \cdot
    \frac{1}{\lambda}} = \int_E |g_n|^p = \|g_n\|_p^p < +\infty \implies
  |g_n|^{p\lambda} \in L^b(E)$.
\end{itemize}

Applichiamo la disuguaglianza di Hölder con
$\dfrac{1}{a} + \dfrac{1}{b} = \mu + \lambda = 1$:
\begin{align*}
  \|g_n\|_r^r 
  &\le \left( \int_E |g_n|^{\mu a} \right)^{1/a} \left( \int_E |g_n|^{p\lambda b} \right)^{1/b}\\[0.5em]
  &= \left( \int_E |g_n| \right)^\mu \cdot \left( \int_E |g_n|^p \right)^\lambda\\[0.5em]
  &= \|g_n\|_1^\mu \cdot \|g_n\|_p^{p\lambda}.
\end{align*}

Sappiamo che $\|g_n\|_1 < \varepsilon$ per $n$ sufficientemente grande
(poiché $f_n \to f$ in $L^1$) e $\|g_n\|_p \le 2$.

Quindi:
\[
  \|g_n\|_r^r \le \varepsilon^\mu \cdot 2^{p(1-\mu)} \implies \lim_{n
    \to \infty}\|g_n\|_r \le \varepsilon^{\mu/r} \cdot 2^{p(1-\mu)/r},
  \quad \forall \epsilon > 0.
\]

Facendo tendere $\varepsilon \to 0$, otteniamo:
\[
  \lim_{n \to \infty} \|g_n\|_r = 0 \iff \lim_{n \to \infty} \|f_n -
  f\|_r = 0.
\]

Quindi $f_n \to f$ in $L^r(E)$.

\subsection*{Parte 3: Convergenza debole in $L^p$}

$f_n \rightharpoonup f$ in $L^p(E)$ (convergenza debole).

Sia $g \in L^q(E)$ con $\dfrac{1}{p} + \dfrac{1}{q} = 1$.

Vogliamo mostrare che:
\[
  \int_E f_n g \to \int_E f g \iff \int_E (f_n - f)g \to 0.
\]

\subsubsection*{Caso particolare: $g \in L^1(E) \cap L^\infty(E)$}

\[
  \left| \int_E (f_n - f)g \right| \le \int_E |f_n - f| |g| \le
  \|g\|_\infty \int_E |f_n - f| = \|g\|_\infty \|f_n - f\|_1
  \xrightarrow[n \to \infty]{} 0.
\]

Quindi:
\[
  \lim_{n \to \infty} \int_E (f_n - f)g = 0 \implies f_n \rightharpoonup
  f \text{ in } L^p(E) \text{ per } g \in L^1 \cap L^\infty.
\]

\subsubsection*{Caso generale: $g \in L^q$}

Per densità di $L^1(E) \cap L^\infty(E)$ in $L^q$, esiste
$h \in L^1(E) \cap L^\infty(E)$ tale che:
\[
  \|g - h\|_q \le \varepsilon \quad \text{per ogni } \varepsilon > 0.
\]

Decomponiamo:
\begin{align*}
  \left| \int_E (f_n - f)g \right| 
  &\le \left| \int_E (f_n - f)(g - h) + (f_n - f)h \right|\\[0.5em]
  &\le \left| \int_E (f_n - f)(g - h) \right| + \left| \int_E (f_n - f)h \right|.
\end{align*}

Per il primo termine, applicando Hölder:
\[
  \le \|f_n - f\|_p \|g - h\|_q + \left| \int_E (f_n - f)h \right| \le 2
  \cdot \varepsilon + \left| \int_E (f_n - f)h \right|,
\]
dove $\|f_n - f\|_p \le 2$.

Quindi:
\[
  \lim_{n \to \infty} \left| \int_E (f_n - f)g \right| \le 2 \cdot
  \varepsilon + \lim_{n \to \infty} \left| \int_E (f_n - f)h \right|.
\]

Il secondo termine tende a 0 (caso precedente). Pertanto:
\[
  \lim_{n \to \infty} \left| \int_E (f_n - f)g \right| \le 2
  \varepsilon, \quad \forall \varepsilon > 0.
\]

Facendo tendere $\varepsilon \to 0$, otteniamo:
\[
  \lim_{n \to \infty} \int_E (f_n - f)g = 0 \implies f_n \rightharpoonup
  f \text{ in } L^p(E).
\]

%%% Local Variables:
%%% mode: LaTeX
%%% TeX-engine: luatex
%%% ispell-local-dictionary: "italian"
%%% TeX-master: "main"
%%% End:

