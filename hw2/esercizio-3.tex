\section*{Esercizio 3}

Consideriamo l'inclusione $\iota : L^q(E) \hookrightarrow L^p(E)$, dove $|E| < \infty$ e $1 \leq p
\le q \leq \infty$. È immediato mostrare che è lineare. Proviamo che è un
operatore continuo mostrando che è limitato.

Sia $f \in L^q(E)$. Applichiamo la disuguaglianza di Hölder con gli
esponenti coniugati $\alpha = q/p$ e $\beta = q/(q-p)$.
\[
  \|f\|_p^p = \int_E |f|^p \cdot 1 \mathrm dx \le \left( \int_E (|f|^p)^{q/p} \right)^{p/q}
  \left( \int_E 1^{\frac{q}{q-p}} \right)^{1 - p/q}.
\]
Pertanto:
\[
  \|\iota f\|_p \le \|f\|_q |E|^{\frac{1}{p} - \frac{1}{q}}.
\]
Quindi
\[
  \left\lVert \iota \right\rVert = \sup_{f \ne 0} \frac{\left\lVert \iota f
    \right\rVert_p}{\left\lVert f \right\rVert_q} \le |E|^{\frac{1}{p} -
    \frac{1}{q}} .
\]
In realtà, scegliendo \(f\) la funzione costante a \(1\), si ha
\(\left\lVert \iota \right\rVert = |E|^{\frac{1}{p} - \frac{1}{q}}\).



%%% Local Variables:
%%% mode: LaTeX
%%% TeX-engine: luatex
%%% ispell-local-dictionary: "italian"
%%% TeX-master: "main"
%%% End:

