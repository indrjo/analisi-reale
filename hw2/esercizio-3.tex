\section*{Esercizio 3}

\subsection*{Obiettivo}
Consideriamo $\iota : L^q(E) \hookrightarrow L^p(E)$ con $|E| < \infty$ e $1 \leq p \le q \leq \infty$.

\subsection*{Parte 1: Continuità}

Sia $f \in L^q(E)$. Applichiamo la disuguaglianza di Hölder con gli
esponenti coniugati $\alpha = q/p$ e $\beta = q/(q-p)$:
\[
  \|f\|_p^p = \int_E |f|^p \cdot 1 \, dx \le \left( \int_E (|f|^p)^{q/p} \right)^{p/q}
  \left( \int_E 1^{\frac{q}{q-p}} \right)^{1 - p/q}.
\]

Pertanto:
\[
  \|f\|_p \le \|f\|_q \cdot |E|^{\frac{1}{p} - \frac{1}{q}}.
\]

L'operatore è limitato, quindi continuo.

\subsection*{Parte 2: Norma dell'operatore}

Si ha $1 \in L^p(E)$ e $1 \in L^q(E)$.

Calcolo delle norme:
\[
  \|f\|_p = |E|^{1/p} \qquad\text{e}\qquad \|f\|_q = |E|^{1/q}.
\]

Quindi:
\[
  \left\|\frac{f}{\|f\|_q} \right\|_p = \frac{\|f\|_p}{\|f\|_q} =
  \frac{|E|^{1/p}}{|E|^{1/q}} = |E|^{1/p - 1/q} < +\infty \implies
  \frac{f}{\|f\|_q} \in L^p(E).
\]

Dalla definizione della norma duale:
\[
  \left\| \frac{f}{\|f\|_q} \right\|_q = \frac{\|f\|_q}{\|f\|_q} = 1.
\]

Quindi:
\[
  \left\| \iota\left(\frac{f}{\|f\|_q} \right)\right\|_p \leq \|\iota\|.
\]

\subsection*{Conclusione}
Testando con la funzione costante $f(x) = 1$, si raggiunge
l'uguaglianza. La norma dell'operatore è quindi esattamente:
\[
  \|\iota\| = |E|^{\frac{1}{p} - \frac{1}{q}}.
\]

%%% Local Variables:
%%% mode: LaTeX
%%% TeX-engine: luatex
%%% ispell-local-dictionary: "italian"
%%% TeX-master: "main"
%%% End:

