\section*{Esercizio 7}
Siano $p\in [1,\infty)$ e $f\in\mathcal{L}^p(\mbb R^n)$ e $\tau_h$ l'operatore lineare da $\mathcal{L}^p(\mbb R^n)$ in $\mathcal{L}^p(\mbb R^n)$ che trasla l'argomento della funzione di un vettore fissato $h\in\mbb R^n$.\\
Come fatti noti prendiamo che $\tau_h$ è un'operatore continuo ed inoltre è un'isometria.\\
Per la densità di $\mathcal{C}^\infty_c(\mbb R^n)$ in $\mathcal{L}^p(\mbb R^n)$ sia una funzione $g\in \mathcal{C}^\infty_c(\mbb R^n)$ che approssimi abbastanza bene $f$, per $g$ i calcoli saranno più semplici dato che $\tau_hg$ e $g$ avranno supporti definitivamente disgiunti.
Abbiamo infatti:
\[
\forall\varepsilon>0, \exists g\in\mathcal{C}^\infty_c(\mbb R^n)\text{ con } \Abs{f-g}_p <\varepsilon
\] 
Per la disuguaglianza triangolare abbiamo ovviamente che:
\[
\abs{\Abs{f}_p-\Abs{g}_p}<\varepsilon
\]
Procediamo ponendo $K=\text{supp}(g)$ e $K_h=\text{supp}(\tau_h g)$, ovviamente $K_h= K+h$, entrambi compatti di $\mbb R^n$, quindi limitati e poiché andremo a studiare il comportamento per $\Abs{h}\rightarrow\infty$ possiamo considerarli definitivamente disgiunti.\\
Calcoliamo ora $\Abs{\tau_h g+g}_p^p$, con $h$ fissato in modo tale che $K\cap K_h=\emptyset$:
\[
\Abs{\tau_h g+g}_p^p=\int_{\mbb R^n} \abs{\tau_h g+g}^p=\int_{K_h} \abs{\tau_h g}^p + \int_{K} \abs{g}^p=\Abs{\tau_h g}_p^p + \Abs{g}_p^p=2\Abs{g}_p^p
\]
Per la seconda uguaglianza usiamo come partizione di $\mbb R^n$ gli insiemi $K, K_h$ e $(K\cup K_h)^c$ unito al fatto che le due funzioni sono nulle al di fuori dei propri supporti, la terza uguaglianza invece per come sono definite le norme e per l'ultima invece il fatto che $\tau_h$ sia un'isometria.

Abbiamo quindi:
$$\lim_{\Abs{h} \to \infty}\Abs{\tau_h g+g}_p=2^{\frac{1}{p}}\Abs{g}_p$$

Possiamo concludere notando che: 
\begin{align*}
\Abs{\tau_h f + f}_p
&= \Abs{\tau_h f - \tau_h g + f - g + \tau_h g + g}_p \\
&\le \Abs{\tau_h f - \tau_h g}_p
   + \Abs{f - g}_p
   + \Abs{\tau_h g + g}_p \\
&\le \Abs{\tau_h g + g}_p + 2\varepsilon .
\end{align*}

Nell'ultima maggiorazione usiamo che $\tau_h$ è lineare e mantiene le norme.

Allora:
\begin{align*}
\limsup_{\Abs{h} \to \infty} \Abs{\tau_h f + f}_p
&\le \limsup_{\Abs{h} \to \infty} \Abs{\tau_h g + g}_p + 2\varepsilon \\
&= 2^{\frac{1}{p}} \Abs{g}_p + 2\varepsilon \\
&\le 2^{\frac{1}{p}} \bigl( \Abs{f}_p + \varepsilon \bigr) + 2\varepsilon \\
&= 2^{\frac{1}{p}} \Abs{f}_p + c_\varepsilon\, \varepsilon .
\end{align*}
Da cui la tesi.
%%% Local Variables:
%%% mode: LaTeX
%%% TeX-engine: luatex
%%% ispell-local-dictionary: "italian"
%%% TeX-master: "main"
%%% End:

