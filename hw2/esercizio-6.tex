\section*{Esercizio 6}

Sia $p\in(1,+\infty)$ e sia
\[
  f_n(x)=1+\sin(n\pi x), \qquad x\in[0,1].
\]
Fissiamo $k\in\mathbb{N}$.

\subsection*{1. Convergenza debole di $f_n^k$}

Poiché $1<p<\infty$, nello spazio $\mathcal{L}^p([0,1])$ la convergenza
debole $g_n\rightharpoonup g$ in $\mathcal{L}^p$ è caratterizzata dalla
condizione
\[
  \int_0^1 g_n(x)\,\varphi(x)\,dx \longrightarrow \int_0^1
  g(x)\,\varphi(x)\,dx \quad \text{per ogni }
  \varphi\in\mathcal{L}^q([0,1]),
\]
dove $q$ è l’esponente coniugato di $p$.

Per il teorema binomiale,
\[
  f_n(x)^k = (1+\sin(n\pi x))^k = \sum_{j=0}^k \binom{k}{j} \sin^j(n\pi
  x).
\]

Ogni potenza $\sin^j(n\pi x)$ può essere scritta come combinazione
lineare finita di termini oscillanti del tipo
\[
  \sin(mn\pi x), \qquad \cos(mn\pi x),
\]
più eventualmente un termine costante (che compare solo se $j$ è pari).

Poiché $\mathcal{L}^q([0,1]) \subset \mathcal{L}^1([0,1])$, per il lemma
di Riemann--Lebesgue si ha, per ogni $\varphi\in \mathcal{L}^q([0,1])$,
\[
  \int_0^1 \varphi(x)\sin(mn\pi x)\,dx \longrightarrow 0, \qquad
  \int_0^1 \varphi(x)\cos(mn\pi x)\,dx \longrightarrow 0, \quad
  \text{per } n\to\infty.
\]

Pertanto tutti i termini oscillanti scompaiono nel limite debole e
rimane soltanto il contributo costante. Per dimostrare quindi che
$f_n^k\rightharpoonup c_k$ per una qualche costante $c_k$ dimostreremo
un fatto equivalente ovvero che:
\[
  \int_0^1 \varphi(x)\cdot(f_n^k-c_k)\rightarrow 0 \qquad
  \text{per}\quad n\to\infty,\forall \varphi\in\mathcal{L}^q([0,1])
\]
Ma per quello che abbiamo detto prima, ovvero che le parti oscillanti di
$f_n^k$ annullano la loro sezione di integrale. Dunque perché il limite
tenda a 0, $c_k$ deve essere il termine noto della sommatoria in
funzioni trigonometriche di primo grado di $f_n^k$, che non dipende
neanche da $n$ poiché è una proprietà indipendente dall'argomento del
seno.  Abbiamo quindi:

\[
  (1+\sin(n\pi x))^k = \sum_{j=0}^k \binom{k}{j}\sin^j(n\pi x).
\]
Per $j$ dispari, la funzione $\sin^j t$ è somma soltanto di altre
funzioni seno, senza termini costanti.  Per $j=2m$ pari, vale la formula
\[
  \sin^{2m} t = \frac{1}{2^{2m}}\binom{2m}{m} + \sum_{\ell=1}^{m}
  a_{\ell}\cos(2\ell t),
\]
per opportuni coefficienti reali $a_{\ell}$.  In particolare, il termine
costante di $\sin^{2m} t$ è
\[
  \frac{1}{2^{2m}}\binom{2m}{m}.
\]
Segue che il termine noto nello sviluppo di $(1+\sin t)^k$ è dato da
\[
  c_k = \sum_{m=0}^{\lfloor k/2\rfloor} \binom{k}{2m} \frac{1}{2^{2m}}
  \binom{2m}{m}.
\]

\subsection*{2. Calcolo di $c_k$ per $k=1,2,3,4$}

Usando la caratterizzazione di $c_k$ calcolata prima si ottiene:

\[
  c_1 = 1 \quad \text{banalmente},
\]

\[
  c_2 = \sum_{m=0}^{\lfloor 1\rfloor} \binom{2}{2m} \frac{1}{2^{2m}}
  \binom{2m}{m} = 1 + \frac{1}{2}=\frac32,
\]

\[
  c_3 = \sum_{m=0}^{\lfloor 3/2\rfloor} \binom{3}{2m} \frac{1}{2^{2m}}
  \binom{2m}{m} = 1 + \frac{3}{2} = \frac52,
\]

\[
  c_4 = \sum_{m=0}^{\lfloor 2\rfloor} \binom{4}{2m} \frac{1}{2^{2m}}
  \binom{2m}{m} = 1 + 3 +\frac{3}{8} = \frac{35}{8}.
\]

\subsection*{3. Mancata compatibilità con la convergenza debole}

Poiché
\[
  f_n \rightharpoonup 1 \quad \text{in } \mathcal{L}^p([0,1]),
\]
se la convergenza debole fosse compatibile con le potenze si dovrebbe
avere
\[
  f_n^k \rightharpoonup 1^k = 1.
\]
Tuttavia, per $k\ge 2$ risulta
\[
  f_n^k \rightharpoonup c_k \neq 1.
\]

Pertanto, in generale,
\[
  f_n \rightharpoonup f \quad \text{non implica} \quad f_n^k
  \rightharpoonup f^k.
\]

%%% Local Variables:
%%% mode: LaTeX
%%% TeX-engine: luatex
%%% ispell-local-dictionary: "italian"
%%% TeX-master: "main"
%%% End:

