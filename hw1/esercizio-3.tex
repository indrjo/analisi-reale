\section{Esercizio 3}

Scelta una qualunque partizione
\(\mathcal{P} := \left\{ a =: a_0 < \dots < a_N := b \right\}\) di
\([a, b]\), abbiamo che
\[
  L(\Gamma_f, \mathcal{P}) \ge \sum_{k=1}^{N} \left\lvert f(a_k) - f(a_{k-1}) \right\rvert =
  V(f, \mathcal{P})
\]
da cui segue facilmente che \(L(\Gamma_f) \ge V(f)\). D'altra parte
\[
  L(\Gamma_f, \mathcal{P}) \le \sum_{k=1}^{N} \left\{ \left\lvert a_k -a_{k-1} \right\rvert
    + \left\lvert f(a_k) - f(a_{k-1}) \right\rvert \right\} = b - a +
  V(f, \mathcal{P})
\]
da cui segue che \(L(\Gamma_f) \le b-a + V(f)\). Possiamo concludere che
\(L(\Gamma_f) < +\infty\) se e solo se \(V(f) < +\infty\).

%%% Local Variables:
%%% mode: LaTeX
%%% TeX-engine: luatex
%%% ispell-local-dictionary: "italian"
%%% TeX-master: "main"
%%% End:

