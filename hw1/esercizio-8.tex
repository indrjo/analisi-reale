\section{Esercizio 8}

\begin{enumerate}[label=(\alph*)]
\item Se \(f_n \to f\) in norma \(\left\lVert \cdot \right\rVert_{BV}\),
  allora \(f_n \to f\) anche in norma
  \(\left\lVert \cdot \right\rVert_\infty\). Quindi \(f = g\).

\item Consideriamo la successione di funzioni
  \begin{align}
    & f_n : [0, 1] \to \mathbb{R} \\
    & f_n (x) := \frac{1}{n} \cos (2 \pi n x)
  \end{align}
  Per l'esercizio 6, si ha che la variazione totale è
  \[
    V(f_n) = \int_{0}^{1} \left\lvert f_n'(x) \right\rvert \mathrm d x = 4
  \]
  Ora \(f_n\) converge in norma
  \(\left\lVert \cdot \right\rVert_\infty\) alla funzione
  \(f : [0, 1] \to \mathbb{R}\) costante a \(0\), ma non è vero che
  \(V(f_n) \to V(f) = 0\) per \(n \to +\infty\).

\item Se esistesse una tale \(c \in \mathbb{R}\), allora considerando la
  successione del punto precedente si avrebbe
  \(\left\lVert f \right\rVert_{\infty, BV} = 4 \leq 0 = c \left\lVert f
  \right\rVert_\infty\).
\end{enumerate}


%%% Local Variables:
%%% mode: LaTeX
%%% TeX-engine: luatex
%%% ispell-local-dictionary: "italian"
%%% TeX-master: "main"
%%% End:

