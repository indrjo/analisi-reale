
\documentclass[11pt, a4paper]{article}

\usepackage{iftex}

\ifpdftex
  \usepackage[utf8]{inputenc}
  \usepackage[T1]{fontenc}
  \usepackage[english,french,italian]{babel}
\else
  \usepackage[no-math]{fontspec}
  \usepackage{polyglossia}
  \setmainlanguage{italian}
  \setotherlanguages{english,french}
\fi

\usepackage{microtype}

\usepackage{enumitem}

\usepackage{libertine}
\usepackage{libertinust1math}
\usepackage{MnSymbol}
\usepackage{mathtools}
\let\underbrace\LaTeXunderbrace
\let\overbrace\LaTeXoverbrace
\usepackage[bb=ams]{mathalfa}

% \usepackage{amsthm}
% \theoremstyle{definition}

% \usepackage{tikz}
\usepackage{pgfplots}
\pgfplotsset{compat=1.18}

\renewcommand\hat\widehat
\renewcommand\tilde\widetilde
\renewcommand\bar\overline
\newcommand\set[1]{\left\{#1\right\}}
\newcommand\abs[1]{\left\lvert#1\right\rvert}
\newcommand\Abs[1]{\left\lVert#1\right\rVert}
\newcommand\mbb[1]{\mathbb #1}
\newcommand\mcal[1]{\mathcal #1}
\newcommand\deriv[2]{\frac{\mathrm d #1}{\mathrm d #2}}
\newcommand\pderiv[2]{\frac{\partial^{#1}}{\partial #2^{#1}}}
% integrale di Lebesgue
\newcommand\leb[3]{\int_{#1} #3 \mathrm d #2}
% note interne
\newcommand\nota[1]{\textcolor{red}{#1}}
\newcommand\Lip{\operatorname{Lip}}



\title{Analisi Reale -- Homeworks 1}
\author{Indrjo Dedej, Micael Defo Noguem, Reema Naz Rasheed, \\
  Umberto Santero Mormile, Lorenzo Torzolini}
\date{Ultimo aggiornamento: \today{}.}

\begin{document}

\maketitle

\section*{Esercizio 1}

Definito lo spazio di Schwartz $\mathcal{S}({\mbb{R}^n})$
\[ \mathcal{S}(\mathbb{R}^n) = \left\{ f \in \mathcal{C}^{\infty}(\mathbb{R}^n) \;:\; \forall\, \alpha, \beta \in \mathbb{N}_0^n,\ \sup_{x \in
      \mathbb{R}^n} \left| x^{\alpha} D^{\beta} f(x) \right| < +\infty \right\}.
\]
Dobbiamo dimostrare che sia chiuso per convoluzioni.

Chiaramente se $f,g\in\mathcal{S}(\mathbb{R}^n) $ allora sono due funzioni $\mathcal{C}^{\infty}$ e quindi la loro convoluzione è ancora $\mathcal{C}^{\infty}$. (Da spiegare?)\\

Fissiamo $\alpha,\beta\in\mbb{N}^n_0$ due multi-indici, avremo quindi:
\[
  \alpha=(\alpha_1,\dots,\alpha_n),\quad \alpha_i\in \mbb{N}_0
\]
\[
  \abs{\alpha}= \alpha_1+\alpha_2 +\dots +\alpha_n
\]
Allo stesso modo per $\beta$. Con $x^\alpha$ si intende:
\[
  x^\alpha\coloneq x_1^{\alpha_1}\cdot x_2^{\alpha_2} \cdot\dots\cdot x_n^{\alpha_n}\cdot
\]
mentre $D^{\beta}$ si intende:
\[
  D^\beta\coloneq\partial_{x_1}^{\beta_1} \cdots \partial_{x_n}^{\beta_n}.
\]
Ora dunque si tratta di mostrare che fissati qualsiasi $\alpha$ e
$\beta$ si ha
che:\[ \sup_{x \in \mathbb{R}^n} \left| x^{\alpha} D^{\beta} (f\ast g)(x) \right| < +\infty
\]

Poiché $f,g$ appartengono ad $\mathcal{S}(\mathbb{R}^n)$ le loro derivate sono maggiorabili
da qualsiasi polinomio, quindi posso invertire derivazione e
integrazione, avremo quindi che:
\[
  \begin{aligned}
    \bigl| x^{\alpha} D^{\beta} (f \ast g)(x) \bigr|
    &= \bigl| x^{\alpha} (D^{\beta} f \ast g)(x) \bigr| \\
    &= \Biggl| \int_{\mathbb{R}^n} x^{\alpha} (D^{\beta} f)(x - y) \, g(y) \, dy \Biggr| \\
    &\leq \int_{\mathbb{R}^n} \bigl| x^{\alpha} (D^{\beta} f)(x - y) \, g(y) \bigr| \, dy
  \end{aligned}
\]
Possiamo maggiorare $\abs{x^\alpha}$ sfruttando la norma euclidea
$\abs{x}$ dei vettori di $\mbb R^n$ infatti
avremo:\[ \abs{x^\alpha}=\abs{x_1^{\alpha_1}\cdot x_2^{\alpha_2} \cdot\dots\cdot x_n^{\alpha_n}}
\]
e, chiaramente: \[ \forall i,\quad \abs{x_i}\leq \abs{x}\leq 1+ \abs{x}
\]
quindi, essendo tutte potenze con esponente naturale
avremo:\[ \abs{x_i}^{\alpha_i}\leq (1+\abs{x})^{\alpha_i}
\]
quindi:
\[
  |x^{\alpha}| = \prod_{i=1}^{n} |x_i|^{\alpha_i} \le \prod_{i=1}^{n} (1 + |x|)^{\alpha_i} = (1 +
  |x|)^{\sum_{i=1}^{n} \alpha_i} = (1 + |x|)^{|\alpha|}.
\]
Utilizzando la disuguaglianza triangolare
\[
  |x| \le |x - y| + |y|
\]
e la proprietà
\[
  (1 + a + b)^N \le (1 + a)^N (1 + b)^N,
\]
ottieniamo la diseguaglianza:
\[
  (1 + |x|)^{|\alpha|} \le (1 + |x - y|)^{|\alpha|} (1 + |y|)^{|\alpha|}.
\]

Inseriamo queste maggiorazioni nell'espressione di interesse:
\[
  \begin{aligned}
    x^{\alpha} D^{\beta} (f \ast g)(x)
    &\le \int_{\mathbb{R}^n} (1 + |x|)^{|\alpha|} |(D^{\beta} f)(x - y)|\, |g(y)| \, dy \\
    &\le \int_{\mathbb{R}^n} 
      \underbrace{(1 + |x - y|)^{|\alpha|} |(D^{\beta} f)(x - y)|}_{A(x - y)} 
      \cdot 
      \underbrace{(1 + |y|)^{|\alpha|} |g(y)|}_{B(y)} 
      \, dy.
  \end{aligned}
\]
\begin{itemize}
\item \textbf{Per il termine $A$:} Poiché
  $f \in \mathcal{S}(\mathbb{R}^n)$, tutte le sue derivate decrescono più velocemente
  dell'inverso di qualsiasi polinomio.  Esiste quindi una costante $C_1$
  (indipendente da $x$ e $y$) tale che
  \[
    \sup_{z \in \mathbb{R}^n} (1 + |z|)^{|\alpha|} |D^{\beta} f(z)| \le C_1.
  \]

\item \textbf{Per il termine $B$:} Poiché
  $g \in \mathcal{S}(\mathbb{R}^n)$, anche la funzione $B(y)$ è a decadimento rapido.  Per
  garantire la convergenza dell'integrale, scegliamo un intero $M$ tale
  che $M > n$ (dimensione dello spazio).  Esiste quindi una costante
  $C_2$ tale che
  \[
    (1 + |y|)^{|\alpha| + M} |g(y)| \le C_2 \quad \Longrightarrow \quad (1 + |y|)^{|\alpha|} |g(y)| \le
    \frac{C_2}{(1 + |y|)^M}.
  \]
\end{itemize}

Sostituendo questi limiti nell'integrale, otteniamo
\[
  \begin{aligned}
    |x^{\alpha} D^{\beta} (f \ast g)(x)|
    &\le \int_{\mathbb{R}^n} A(x - y) \, B(y) \, dy \\
    &\le \int_{\mathbb{R}^n} C_1 \cdot \frac{C_2}{(1 + |y|)^M} \, dy \\
    &= C_1 C_2 \int_{\mathbb{R}^n} \frac{1}{(1 + |y|)^M} \, dy.
  \end{aligned}
\]

Poiché abbiamo scelto $M > n$, l'integrale
$\int_{\mathbb{R}^n} (1 + |y|)^{-M} \, dy$ converge. Pertanto, la quantità
$ |x^{\alpha} D^{\beta} (f \ast g)(x)| $ è limitata uniformemente in $x$.



%%% Local Variables:
%%% mode: LaTeX
%%% TeX-engine: luatex
%%% ispell-local-dictionary: "italian"
%%% TeX-master: "main"
%%% End:


\section{Esercizio 2}

%%% Local Variables:
%%% mode: LaTeX
%%% TeX-engine: luatex
%%% ispell-local-dictionary: "italian"
%%% TeX-master: "main"
%%% End:


\section*{Esercizio 3}

Verifichiamo che
\(\left\langle \cdot, \cdot \right\rangle : \mathcal{H} \times \mathcal{H} \to K\) (dove
\(K\) è \(\mathbb{R}\) oppure \(\mathbb{C}\)) è un prodotto scalare.
\begin{enumerate}
\item Linearità nel primo argomento. Siano \(f, g, h \in \mathcal{H}\) e
  \(\lambda_1, \lambda_2 \in K\).
  \begin{align*}
    \left\langle \lambda_1 f + \lambda_2 g, h \right\rangle
    &= \int_{[0,1]} (\lambda_1 f + \lambda_2 g)'(x) h'(x) \mathrm d x = \\
    &= \lambda_1 \int_{[0,1]}f'(x) h'(x) \mathrm d x + \lambda_2 \int_{[0,1]}g'(x) h'(x)
      \mathrm d x = \\
    &=  \lambda_1 \left\langle f, h \right\rangle +  \lambda_2 \left\langle g, h \right\rangle .
  \end{align*}
\item Skew-simmetria. In realtà è proprio simmetrico, cioè
  \(\left\langle f, g \right\rangle = \left\langle g, f \right\rangle\) per ogni \(f, g \in \mathcal{H}\).
\item Sia \(f \in \mathcal{H}\). Allora
  \(\left\langle f, f \right\rangle = \left\lVert f' \right\rVert_2^2 \ge 0\). Così
  facendo, si deduce che se \(\left\langle f, f \right\rangle = 0\), allora,
  \(f' = 0\) quasi ovunque. Poiché \(f \in AC[0, 1]\), si ha che
  \[
    f(x) = f(0) + \int_{0}^{x} f'(t) \mathrm d t = 0 \qquad\text{per ogni } x \in
    [0, 1] .
  \]
\end{enumerate}

Rimane da provare che lo spazio vettoriale \(\mathcal{H}\) con la norma indotta
dal prodotto scalare
\[
  \left\lVert f \right\rVert := \sqrt{\left\langle f, f \right\rangle}
\]
è uno spazio di Banach. Sia quindi
\(\left\{ f_n \mid n \in \mathbb{N} \right\} \subseteq \mathcal{H}\) una successione di Cauchy rispetto
alla norma \(\left\lVert \cdot \right\rVert\). Se è così, allora
\(\left\{ f_n' \mid n \in \mathbb{N} \right\} \subseteq \mathcal{L}^2[0, 1]\) è una successione di
Cauchy rispetto alla norma \(\left\lVert \cdot \right\rVert_2\). Infatti
ricordare che
\[
  \left\lVert f \right\rVert = \left\lVert f' \right\rVert_2 .
\]
Poiché
\(\left(\mathcal{L}^2[0, 1], \left\lVert \cdot \right\rVert_2\right)\) è di Banach,
allora \(f_n'\) converge a un qualche elemento di
\(\mathcal{L}^2[0, 1]\) che indichiamo con \(f'\). A questo punto introduciamo
\begin{align*}
  & f : [0, 1] \to K \\
  & f(x) := \int_{0}^{x} f'(t) \mathrm d t .
\end{align*}
Questa funzione è assolutamente continua, \(f(0) = 0\) e la sua derivata
quasi ovunque è proprio \(f'\): pertanto \(f \in \mathcal{H}\). Concludiamo
\[
  \left\lVert f - f_n \right\rVert = \left\lVert (f - f_n)'
  \right\rVert_2 = \left\lVert f' - f_n' \right\rVert_2 \to 0 \quad \text{per
  } n \to +\infty .
\]

%%% Local Variables:
%%% mode: LaTeX
%%% TeX-engine: luatex
%%% ispell-local-dictionary: "italian"
%%% TeX-master: "main"
%%% End:


\section*{Esercizio 4}

\subsection*{Ipotesi}
Sia $\{f_n\}$ limitata in $L^p(E)$ ($p \in (1, +\infty)$) e convergente
fortemente a $f$ in $L^1(E)$. Esiste allora una sottosuccessione
convergente quasi ovunque.

\subsection*{Parte 1: $f \in L^p(E)$}

Si ha $\displaystyle\lim_{k \to \infty} f_{n_k}(x) = f(x)$ quasi ovunque, il
che implica:
\[
  \lim_{k \to \infty} |f_{n_k}(x)| = |f(x)|.
\]

Poiché $x \mapsto |x|^p$ è continua:
\[
  \lim_{k \to \infty} |f_{n_k}(x)|^p = |f(x)|^p.
\]

Dal lemma di Fatou:
\[
  \int_E |f|^p \leq \int_E \liminf_{k \to \infty} |f_{n_k}|^p \le \liminf_{k \to \infty}
  \int_E |f_{n_k}|^p \le 1.
\]

Quindi $f \in L^p(E)$.

\subsection*{Parte 2: Convergenza forte in $L^r$ ($1 < r < p$)}

Sappiamo che $f_n, f \in L^1(E) \cap L^p(E)$ e:
\[
  \|f_n - f\|_p \leq \|f_n\|_p + \|f\|_p \leq 2.
\]

Poniamo $g_n = f_n - f$.

Poiché $L^1(E)$ e $L^p(E)$ sono spazi vettoriali, si ha
$g_n \in L^1(E) \cap L^p(E)$.

Da $g_n \in L^1 \cap L^p$, esistono $\mu, \lambda > 0$ tali che
$\mu + \lambda = 1$ e $r = 1 \cdot \mu + p \cdot \lambda$.

Quindi:
\[
  \|g_n\|_r^r = \int_E |g_n|^r = \int_E |g_n|^\mu \cdot
  |g_n|^{p\lambda}.
\]

Poniamo $a = \dfrac{1}{\mu} > 1$ e $b = \dfrac{1}{\lambda} > 1$.

Verifichiamo:
\begin{itemize}[leftmargin=2em]
\item
  $\displaystyle\int_E |g_n|^{\mu a} = \int_E |g_n|^{\mu \cdot
    \frac{1}{\mu}} = \int_E |g_n| = \|g_n\|_1 < +\infty \implies
  |g_n|^\mu \in L^a(E)$.
    
\item
  $\displaystyle\int_E |g_n|^{p\lambda b} = \int_E |g_n|^{p\lambda \cdot
    \frac{1}{\lambda}} = \int_E |g_n|^p = \|g_n\|_p^p < +\infty \implies
  |g_n|^{p\lambda} \in L^b(E)$.
\end{itemize}

Applichiamo la disuguaglianza di Hölder con
$\dfrac{1}{a} + \dfrac{1}{b} = \mu + \lambda = 1$:
\begin{align*}
  \|g_n\|_r^r 
  &\le \left( \int_E |g_n|^{\mu a} \right)^{1/a} \left( \int_E |g_n|^{p\lambda b} \right)^{1/b}\\[0.5em]
  &= \left( \int_E |g_n| \right)^\mu \cdot \left( \int_E |g_n|^p \right)^\lambda\\[0.5em]
  &= \|g_n\|_1^\mu \cdot \|g_n\|_p^{p\lambda}.
\end{align*}

Sappiamo che $\|g_n\|_1 < \varepsilon$ per $n$ sufficientemente grande
(poiché $f_n \to f$ in $L^1$) e $\|g_n\|_p \le 2$.

Quindi:
\[
  \|g_n\|_r^r \le \varepsilon^\mu \cdot 2^{p(1-\mu)} \implies \lim_{n
    \to \infty}\|g_n\|_r \le \varepsilon^{\mu/r} \cdot 2^{p(1-\mu)/r},
  \quad \forall \epsilon > 0.
\]

Facendo tendere $\varepsilon \to 0$, otteniamo:
\[
  \lim_{n \to \infty} \|g_n\|_r = 0 \iff \lim_{n \to \infty} \|f_n -
  f\|_r = 0.
\]

Quindi $f_n \to f$ in $L^r(E)$.

\subsection*{Parte 3: Convergenza debole in $L^p$}

$f_n \rightharpoonup f$ in $L^p(E)$ (convergenza debole).

Sia $g \in L^q(E)$ con $\dfrac{1}{p} + \dfrac{1}{q} = 1$.

Vogliamo mostrare che:
\[
  \int_E f_n g \to \int_E f g \iff \int_E (f_n - f)g \to 0.
\]

\subsubsection*{Caso particolare: $g \in L^1(E) \cap L^\infty(E)$}

\[
  \left| \int_E (f_n - f)g \right| \le \int_E |f_n - f| |g| \le
  \|g\|_\infty \int_E |f_n - f| = \|g\|_\infty \|f_n - f\|_1
  \xrightarrow[n \to \infty]{} 0.
\]

Quindi:
\[
  \lim_{n \to \infty} \int_E (f_n - f)g = 0 \implies f_n \rightharpoonup
  f \text{ in } L^p(E) \text{ per } g \in L^1 \cap L^\infty.
\]

\subsubsection*{Caso generale: $g \in L^q$}

Per densità di $L^1(E) \cap L^\infty(E)$ in $L^q$, esiste
$h \in L^1(E) \cap L^\infty(E)$ tale che:
\[
  \|g - h\|_q \le \varepsilon \quad \text{per ogni } \varepsilon > 0.
\]

Decomponiamo:
\begin{align*}
  \left| \int_E (f_n - f)g \right| 
  &\le \left| \int_E (f_n - f)(g - h) + (f_n - f)h \right|\\[0.5em]
  &\le \left| \int_E (f_n - f)(g - h) \right| + \left| \int_E (f_n - f)h \right|.
\end{align*}

Per il primo termine, applicando Hölder:
\[
  \le \|f_n - f\|_p \|g - h\|_q + \left| \int_E (f_n - f)h \right| \le 2
  \cdot \varepsilon + \left| \int_E (f_n - f)h \right|,
\]
dove $\|f_n - f\|_p \le 2$.

Quindi:
\[
  \lim_{n \to \infty} \left| \int_E (f_n - f)g \right| \le 2 \cdot
  \varepsilon + \lim_{n \to \infty} \left| \int_E (f_n - f)h \right|.
\]

Il secondo termine tende a 0 (caso precedente). Pertanto:
\[
  \lim_{n \to \infty} \left| \int_E (f_n - f)g \right| \le 2
  \varepsilon, \quad \forall \varepsilon > 0.
\]

Facendo tendere $\varepsilon \to 0$, otteniamo:
\[
  \lim_{n \to \infty} \int_E (f_n - f)g = 0 \implies f_n \rightharpoonup
  f \text{ in } L^p(E).
\]

%%% Local Variables:
%%% mode: LaTeX
%%% TeX-engine: luatex
%%% ispell-local-dictionary: "italian"
%%% TeX-master: "main"
%%% End:


\section*{Esercizio 5}

Consideriamo il sottospazio
\[
  V := \left\{ p : [-1, 1] \to \mathbb{R} \mid p \text{ funzione polinomiale di grado
    } \leq N \right\} .
\]
È un sottospazio lineare chiuso (i sottospazi di dimensione finita di
uno spazio normato sono tutti chiusi.) Possiamo quindi usare il teorema
di proiezione: esiste uno e un solo \(p_f \in V\) tale che
\(\left\lVert f - p_f \right\rVert_2 = \inf_{p \in V} \left\lVert f - p
\right\rVert_2\).

Sia quindi \(f \in \mathcal{L}^2[-1, 1]\) e
\(p : [-1, 1] \to \mathbb{R}\) la funzione polinomiale
\[
  p(x) := \sum_{k=0}^{N} a_kx^k
\]
e calcoliamo la distanza da \(f\):
\begin{align*}
  \left\lVert f - p \right\rVert_2^2
  &= \int_{-1}^{1} (f(x) - p(x))^2 \mathrm d x = \\
  &= \int_{-1}^{1} p(x)^2 \mathrm d x + \int_{-1}^{1} f(x)^2 \mathrm d x - 2
    \sum_{k=0}^{N} a_k \int_{-1}^{1} x^k f(x) \mathrm d x 
\end{align*}
Troviamo i coefficienti \(a_0, \dots, a_N \in \mathbb{R}\) che minimizzano la
distanza sopra, i quali sono unicamente determinati come osservato
all'inizio. Derivando rispetto ad \(a_\mu\) e ponendo uguale a \(0\) si
trova il sistema
\[
  \sum_{j=0}^{N} a_j \int_{-1}^{1} x^{\mu + j} \mathrm d x = \int_{-1}^{1} x^\mu f(x)
  \mathrm d x \quad\text{per } \mu \in \left\{ 0, \dots, N \right\}
\]
nelle \(N\) incognite \(a_1, \dots, a_N\).

Osserviamo inoltre che in alcuni casi il sistema è più semplice da
risolvere. Se \(f : [-1, 1] \to \mathbb{R}\) è pari, allora la funzione polinomiale
\begin{align*}
  & q : [-1, 1] \to \mathbb{R} \\
  & q(x) := p_f(-x)
\end{align*}
soddisfa
\(\left\lVert f - q \right\rVert_2 = \left\lVert f - p_f
\right\rVert_2\). Questo però implica che \(p_f = q\) a causa
dell'unicità. Quindi \(p_f\) è pari.

Le funzioni polinomiali pari sono quelle i cui coefficienti dei monomi
di grado dispari sono nulli. Questo nel nostro caso significa rimangono
da determinare gli \(a_k\) con \(k\) pari.



%%% Local Variables:
%%% mode: LaTeX
%%% TeX-engine: luatex
%%% ispell-local-dictionary: "italian"
%%% TeX-master: "main"
%%% End:


\section*{Esercizio 6}

Sia $\mathcal{H} = L^2(\mathbb{R}, \mathbb{C})$. Si consideri l'operatore $T \in \mathcal{L}(\mathcal{H})$ definito da:
\[ (Tf)(x) = x \chi_{[0,1]}(x) f(x) \]
Indichiamo $h(x) = x \chi_{[0,1]}(x)$. $T$ è quindi l'operatore di moltiplicazione per la funzione $h$.

\paragraph{1. Norma di $T$:}
Per un operatore di moltiplicazione $M_h$ su $L^2$, la norma dell'operatore è data dalla norma infinito essenziale della funzione moltiplicatrice:
\[ \|T\| = \|h\|_{\infty} \]
Qui, $h(x)$ è definita da:
\[ h(x) = 
\begin{cases} 
x & \text{se } x \in [0, 1] \\
0 & \text{altrimenti}
\end{cases}
\]
Sull'intervallo $[0, 1]$, il valore massimo di $|x|$ è $1$. Al di fuori, la funzione è nulla. Pertanto:
\[ \|T\| = \sup_{x \in [0,1]} |x| = 1 \]

\paragraph{2. Spettro $\sigma(T)$:}

Lo spettro di un operatore di moltiplicazione per una funzione $h$ corrisponde all'immagine essenziale di tale funzione. 
L'immagine della funzione $h(x) = x \chi_{[0,1]}(x)$ è l'insieme dei valori raggiunti da $h$ su $\mathbb{R}$:
\begin{itemize}
    \item Per $x \in [0, 1]$, $h(x)$ percorre l'intervallo $[0, 1]$.
    \item Per $x \notin [0, 1]$, $h(x) = 0$, valore già incluso nell'intervallo precedente.
\end{itemize}
Lo spettro è quindi l'intervallo chiuso:
\[ \sigma(T) = [0, 1] \]

\paragraph{3. Spettro puntuale $\sigma_p(T)$:}
$\lambda$ è un autovalore di $T$ se esiste una funzione $f \in L^2(\mathbb{R})$ non nulla tale che $Tf = \lambda f$, ovvero:
\[ (h(x) - \lambda) f(x) = 0 \quad \text{quasi ovunque} \]
Questo impone che $f$ sia nulla quasi ovunque al di fuori dell'insieme $E_\lambda = \{x \in \mathbb{R} : h(x) = \lambda\}$. Affinché $f$ sia non nulla in $L^2$, la misura di Lebesgue di $E_\lambda$ deve essere strettamente positiva ($\mu(E_\lambda) > 0$).

\begin{itemize}
    \item \textbf{Caso $\lambda \in (0, 1]$:} L'equazione $h(x) = \lambda$ possiede un'unica soluzione $x = \lambda$. Un punto singolo ha misura nulla. Pertanto, questi valori non sono autovalori.
    \item \textbf{Caso $\lambda = 0$:} L'insieme $E_0 = \{x \in \mathbb{R} : h(x) = 0\}$ è uguale a $(-\infty, 0] \cup (1, +\infty)$. Questo insieme ha misura infinita (quindi $>0$). Possiamo definire una funzione $f = \chi_{[2,3]} \in L^2$ che soddisfa $Tf = 0$. 
\end{itemize}
L'unico autovalore dell'operatore è quindi:
\[ \sigma_p(T) = \{0\} \]

%%% Local Variables:
%%% mode: LaTeX
%%% TeX-engine: luatex
%%% ispell-local-dictionary: "italian"
%%% TeX-master: "main"
%%% End:


\section*{Esercizio 7}

% \begin{enumerate}
% \item {\em Mostriamo che \(T \in \mathcal{L}(\mathcal{H})\) è autoaggiunto.} In questo caso,
%   ricordiamo che il prodotto scalare di \(\mathcal{H}\) è definito come
%   \[
%     \left\langle f, g \right\rangle = \int_{[0,1]}^{} f(x) \bar{g(x)} \mathrm d x .
%   \]
%   Verifichiamo quindi che
%   \(\left\langle Tf, g \right\rangle = \left\langle f, Tg \right\rangle\) per calcolo
%   diretto. Il primo membro è
%   \[
%     \left\langle Tf, g \right\rangle = \int_{[0, 1]}^{} \int_{[0, 1]}^{} K(x, y) f(y)
%     \bar{g(x)} \mathrm d y \mathrm d x
%   \]
%   mentre il secondo membro è
%   \[
%     \left\langle f, Tg \right\rangle = \int_{[0, 1]}^{} \int_{[0, 1]}^{} \bar{K(x, y)}
%     f(y) \bar{g(x)} \mathrm d y \mathrm d x
%   \]
%   Per concludere questa parte basta osservare che \(K = \bar{K}\).

% \item {\em \(T\) è compatto.}
%   \(K \in \mathcal{L}^2([0, 1] \times [0, 1])\) e quindi \(T\) è un operatore di
%   Hilbert-Schmidt, che è compatto. (Cfr. Proposizione 3.5.2 delle note
%   del corso.)

% \item {\em Trovare lo spettro di \(T\).} Osserviamo che essendo \(T\)
%   autoaggiunto, allora i suoi autovalori sono tutti reali
%   (cfr. Proposizione 3.6.6). Inoltre gli elementi non nulli dello spetto
%   sono autovalori di \(T\) (cfr. Proposizione 3.6.9). {\color{red} [Niente.]}
% \end{enumerate}

Sia $\mathcal{H} = L^2([0, 1], \mathbb{C})$. Si considerino $A, B \in L^2([0, 1], \mathbb{R})$ due funzioni linearmente indipendenti. Si definisce l'operatore $T \in \mathcal{L}(\mathcal{H})$ da:
\[ (Tf)(x) = \int_{0}^{1} K(x, y) f(y) \, dy \quad \text{con } K(x, y) = A(x)B(y) + A(y)B(x) \]
\\
Per un operatore integrale, il carattere autoaggiunto è garantito se il nucleo soddisfa
\\ $K(x, y) = \overline{K(y, x)}$. 
Qui, le funzioni $A$ et $B$ sono reali, quindi $K$ è reale. Infatti:
\[ K(y, x) = A(y)B(x) + A(x)B(y) = A(x)B(y) + A(y)B(x) = K(x, y) \]
Essendo il nucleo reale e simmetrico, l'operatore $T$ è autoaggiunto $T = T^*$.

\paragraph{1. Compattezza:}
Si ha:
\[ (Tf)(x) = A(x) \int_{0}^{1} B(y)f(y) \, dy + B(x) \int_{0}^{1} A(y)f(y) \, dy \]
Utilizzando il prodotto scalare standard $\langle f, g \rangle = \int f \bar{g}$, si ha:
\[ (Tf)(x) = \langle f, B \rangle A(x) + \langle f, A \rangle B(x) \]
Questo mostra che $\text{Im}(T) \subseteq \text{Vect}(A, B)$. L'immagine di $T$ è quindi di dimensione al massimo 2. 
$T$ è un operatore di rango finito, il che implica direttamente che è compatto.

\paragraph{2. Spettro $\sigma(T)$:}
Poiché $T$ è compatto in dimensione infinita, il suo spettro è composto dai suoi autovalori e da $0$. 
Sia $\lambda \neq 0$ un autovalore. Allora $f = \frac{1}{\lambda} Tf \in \text{Vect}(A, B)$. Poniamo $f = \alpha A + \beta B$. 
L'equazione $Tf = \lambda f$ porta al sistema lineare:
\[
\begin{cases}
\alpha \langle A, B \rangle + \beta \|B\|^2 = \lambda \alpha \\
\alpha \|A\|^2 + \beta \langle A, B \rangle = \lambda \beta
\end{cases}
\implies
\begin{pmatrix} \langle A, B \rangle - \lambda & \|B\|^2 \\ \|A\|^2 & \langle A, B \rangle - \lambda \end{pmatrix} \begin{pmatrix} \alpha \\ \beta \end{pmatrix} = 0
\]
Il determinante deve essere nullo: $(\langle A, B \rangle - \lambda)^2 - \|A\|^2 \|B\|^2 = 0$. 
Gli autovalori sono $\lambda_{\pm} = \langle A, B \rangle \pm \|A\| \|B\|$. 
Pertanto, $\sigma(T) = \{0, \langle A, B \rangle + \|A\| \|B\|, \langle A, B \rangle - \|A\| \|B\|\}$.

\paragraph{3. $\|K\|_{L^2} > \|T\|$:}
1. La norma dell'operatore è
$\|T\| = \max(|\lambda_+|, |\lambda_-|) = \|A\|\|B\| + |\langle A, B \rangle|$.
\\
2. La norma di Hilbert-Schmidt è
$\|K\|_{L^2}^2 = \iint |K(x,y)|^2 dx dy = 2\|A\|^2\|B\|^2 + 2\langle A, B \rangle^2$.
\\
Si nota che
$\|K\|_{L^2}^2 - \|T\|^2 = (\|A\|\|B\| - |\langle A, B \rangle|)^2$.  In base alla
diseguaglianza di Cauchy-Schwarz, poiché $A$ e $B$ sono linearmente
indipendenti, $|\langle A, B \rangle| < \|A\|\|B\|$.  La differenza è quindi
strettamente positiva: $\|K\|_{L^2} > \|T\|$. Le quantità non coincidono.



%%% Local Variables:
%%% mode: LaTeX
%%% TeX-engine: luatex
%%% ispell-local-dictionary: "italian"
%%% TeX-master: "main"
%%% End:


\section{Esercizio 8}

%%% Local Variables:
%%% mode: LaTeX
%%% TeX-engine: luatex
%%% ispell-local-dictionary: "italian"
%%% TeX-master: "main"
%%% End:



\end{document}

%%% Local Variables:
%%% mode: LaTeX
%%% TeX-engine: luatex
%%% ispell-local-dictionary: "italian"
%%% TeX-master: t
%%% End:

