\section*{Esercizio 4}

L'idea per la risoluzione di questo esercizio è quella di sfruttare la caratterizzazione degli insiemi di misura nulla per la misura di Lebesgue sulla retta reale, ovvero:\begin{center}
    $\forall E \subset \mathbb{R}, |E|=0 \Rightarrow \forall \varepsilon >0, \exists \{I_n=(x_n,y_n)\}_{n\in\mathbb{N}}$ t.c. $E\subset \bigcup\limits_{n\in\mathbb{N}} I_n$ e $\sum\limits_
    {n\in\mathbb{N}} |I_n| \leq \varepsilon$
\end{center}
Quindi un insieme ha misura nulla se e solo se è possibile ricoprirlo con una quantità numerabile di intervalli piccoli a piacere.\\

Passiamo ora alla soluzione: sia $\varepsilon>0$ arbitrario. 
Per definizione di assoluta continuità esiste $\delta=\delta(\varepsilon)>0$ tale che per ogni collezione finita di intervalli disgiunti $\{[x_k,y_k]\}_{k=1}^N$ con
\[
\sum_{k=1}^N (y_k-x_k) < \delta
\]
vale
\[
\sum_{k=1}^N |f(y_k)-f(x_k)| < \varepsilon.
\]

Utilizziamo il $\delta$ della definizione di assoluta continuità per ottenere un ricoprimento di $E$ tramite intervalli di misura piccola, infatti poichè $|E|=0$ esiste una famiglia numerabile di intervalli aperti \(\{I_j\}_{j=1}^{\infty}\) tale che
\[
E \subset \bigcup_{j=1}^{\infty} I_j \quad\text{e}\quad \sum_{j=1}^{\infty} |I_j| < \delta.
\]

Ora suddividiamo eventualmente ciascun intervallo \(I_j\) in sottointervalli \(\tilde{I_{j,l}}\) dove $f$ è monotona. Questo è sempre possibile grazie alla continuità di \(f\).
Applichiamo questa ulteriore divisione perché così possiamo sfruttare la proprietà dell'assoluta continuità che controlla la variazione negli estremi di ogni intervallo.

Quindi $\{\tilde{I_{j,l}}\}_{j,l\in \mathbf{N}}$ è un ricoprimento aperto di $E$ di misura totale minore di $\delta$, dunque per ogni sua collezione finita di elementi la misura sarà ancora minore di $\delta$, concludiamo allora per l'assolutà continuità di $f$ che la misura dell'immagine degli intervalli sarà minore di $\varepsilon$.

Poiché $\{\tilde{I_{j,l}}\}_{j,l\in \mathbf{N}}$ è un ricoprimento di $E$ allora $\{f(\tilde{I_{j,l}})\}_{j,l\in \mathbf{N}}$ è un ricoprimento di $f(E)$ e la cui misura è la serie delle misure degli intervalli a cui è applicata $f$, serie a termini positivi di cui ogni somma parziale è limitata da $\varepsilon$ stesso, dunque la serie è maggiorata da quest'ultimo.

Poiché $\varepsilon>0$ era arbitrario, concludiamo che $|f(E)|=0$.


%%% Local Variables:
%%% mode: LaTeX
%%% TeX-engine: luatex
%%% ispell-local-dictionary: "italian"
%%% TeX-master: "main"
%%% End:

