\section*{Esercizio 7}

\begin{enumerate}[label=(\alph*)]
\item Mostriamo che \(BV[0, 1]\) è uno spazio vettoriale con l'addizione
  e la moltiplicazione per scalare definite puntualmente.  Siano
  \(f, g \in BV[0, 1]\) e \(\alpha \in \mathbb{R}\). Prendiamo anche una qualsiasi
  partizione \(\mathcal{P} := \set{ 0 =: a_0 < \dots < a_N := 1 }\). Allora
  \[
    V(\alpha f, \mathcal{P}) = \sum_{k=1}^{N} \left\lvert \alpha f(a_k) - \alpha f(a_{k-1})
    \right\rvert = \left\lvert \alpha \right\rvert V(f, \mathcal{P})
  \]
  quindi segue immediatamente che anche \(\alpha f \in BV[0, 1]\) e
  \(V(\alpha f) = \alpha V(f)\). Inoltre
  \[
    V(f+g, \mathcal{P}) = \sum_{k=1}^{N} \left\lvert f(a_k) + g(a_k) - f(a_{k-1}) -
      g(a_{k-1}) \right\rvert \leq V(f, \mathcal{P}) + V(g, \mathcal{P})
  \]
  per cui \(f+g \in BV[0, 1]\).

  Adesso vediamo come
  \(\left\lVert \cdot \right\rVert_{\infty,BV} : BV[0, 1] \to \mathbb{R}\) è una norma. Per
  quanto abbiamo appena visto ed essendo
  \(\left\lVert \cdot \right\rVert_\infty\) una norma, sono automatiche la
  disuguaglianza triangolare e la proprietà di omogeneità. Inoltre è
  immediato verificare che
  \(\left\lVert f \right\rVert_{\infty,BV} \geq 0\) per ogni
  \(f \in BV[0, 1]\). Supponiamo ora che
  \(\left\lVert f \right\rVert_{\infty,BV} = 0\): allora in particolare
  \(\left\lVert f \right\rVert_\infty = 0\), da cui segue che \(f = 0\).

\item Sia \(\set{f_n \mid n \in \mathbb{N} \subseteq BV[0, 1]}\) una successione di Cauchy. In
  particolare, la successione la successione è limitata, cioè possiamo
  scegliere \(M > 0\) tale che
  \(\left\lVert f_n \right\rVert_{\infty,BV} \le M\) per ogni
  \(n \in \mathbb{N}\). Inoltre la successione è di Cauchy anche rispetto alla
  norma \(\left\lVert \cdot \right\rVert_\infty\). Osserviamo a tal proposito che
  le \(f_n \in BV[0, 1]\) sono limitate, e quindi
  \(\set{f_n \mid n \in \mathbb{N}} \subseteq B[0, 1]\): questo spazio vettoriale con la norma
  \(\left\lVert \cdot \right\rVert_\infty\) è uno spazio di Banach. Quindi
  \(f_n\) converge ad \(f \in B[0, 1]\) definita come segue:
  \[
    f(x) := \lim_{n \to +\infty} f_n(x) .
  \]
  Mostreremo che
  \begin{enumerate}[label=(\roman*)]
  \item \(f \in BV[0, 1]\).
  \item \(\lim_{n \to +\infty} \left\lVert f_n - f \right\rVert_{\infty,BV} = 0\).
  \end{enumerate}
  \begin{enumerate}[label=(\roman*)]
  \item Sia \(\mathcal{P} := \set{ 0 =: a_0 < \dots < a_N := 1 }\) una partizione
    di \([a, b]\): allora
    \begin{align*}
      V(f, \mathcal{P}) &= \sum_{k=1}^{N} \left\lvert f(a_k) - f(a_{k-1}) \right\rvert =
      \\
              & = \lim_{n \to +\infty} \sum_{k=1}^{N} \left\lvert f_n(a_k) - f_n(a_{k-1})
                \right\rvert \leq \\
              & \leq \lim_{n \to +\infty} V(f_n) \le M .
    \end{align*}
  \item Sia \(\epsilon > 0\). Poiché la successione è di Cauchy rispetto a
    \(\left\lVert \cdot \right\rVert_{\infty,BV}\), sia \(\bar n\) tale che
    \(\left\lVert f_m - f_n \right\rVert_{\infty,BV} \le \epsilon\) per ogni
    \(m , n \ge \bar n\).
    \begin{align*}
      V(f - f_n, \mathcal{P}) &= \sum_{k=1}^{N} \left\lvert (f - f_n)(a_k) - (f -
                      f_n)(a_{k-1}) \right\rvert = \\
                    &= \lim_{m \to +\infty} \sum_{k=1}^{N} \left\lvert (f_m - f_n)(a_k) - (f_m -
                      f_n)(a_{k-1}) \right\rvert \leq \\
                    &\leq \lim_{m \to +\infty} V(f_m - f_n)
    \end{align*}
    Allora per \(n \geq \bar n\) abbiamo
    \[
      V(f-f_n) \le \lim_{m \to +\infty} V(f_m - f_n) \le \lim_{m \to +\infty} \left\lVert
        f_m - f_n \right\rVert_{\infty,BV} \le \epsilon .
    \]
    Pertanto \(V(f-f_n) \to 0\) per \(n \to +\infty\) e quindi possiamo
    concludere.
  \end{enumerate}

\item Ricordiamo che se \(f \in AC[0, 1]\), allora
  \[
    V(f) = \int_{0}^{1}\left\lvert f' \right\rvert = \left\lVert f'
    \right\rVert_1 .
  \]
  Sia \(\set{ f_n \mid n \in \mathbb{N} } \subseteq AC[0, 1]\) convergente a qualche
  \(f \in BV[0, 1]\) in norma
  \(\left\lVert \cdot \right\rVert_{\infty,BV}\) e mostriamo che deve essere
  \(f \in AC[0, 1]\). In particolare per ogni \(n \in \mathbb{N}\) esistono quasi
  ovunque le derivate \(f_n' \in L^1[0, 1]\) si ha
  \[
    f_n(x) = f_n(0) + \int_{0}^{x}f_n' \quad\text{per quasi ogni } x \in [0, 1] .
  \]
  Poiché una successione convergente converge secondo Cauchy, in norma
  \(\left\lVert \cdot \right\rVert_\infty\) la successione
  \(\set{ f_n \mid n \in \mathbb{N} }\) e in norma
  \(\left\lVert \cdot \right\rVert_1\) la successione
  \(\set{ f_n' \mid n \in \mathbb{N} }\) sono di Cauchy. Come sopra,
  \(f_n\) converge in norma \(\left\lVert \cdot \right\rVert_\infty\) al limite
  puntuale di \(f_n\) che indichiamo con \(f\). Anche \(L^1[0, 1]\) è
  completo, quindi \(f_n' \to g \in L^1[0, 1]\) in norma
  \(\left\lVert \cdot \right\rVert_1\). Introduciamo la funzione
  assolutamente continua
  \begin{align*}
    & f^\ast : [0, 1] \to \mathbb{R} \\
    & f^\ast (x) := f(0) + \int_{0}^{x}g 
  \end{align*}
  Osserviamo che
  \begin{align*}
    \left\lVert f^\ast - f_n \right\rVert_\infty
    &= \sup_{x \in [0, 1]} \left\lvert f(0) + \int_{0}^{x}g - f_n(0) -
      \int_{0}^{x}f_n' \right\rvert \le \\
    &\le \left\lvert f(0) - f_n(0) \right\rvert + \sup_{x \in [0, 1]}
      \left\lvert \int_{0}^{x} (g - f_n') \right\rvert \le \\
    &\le \left\lvert f(0) - f_n(0) \right\rvert + \sup_{x \in [0, 1]}
      \int_{0}^{x} \left\lvert g - f_n' \right\rvert \le \\
    &\le \left\lVert f - f_n \right\rVert_\infty + \int_{0}^{1} \left\lvert g
      -f_n' \right\rvert = \\
    &= \left\lVert f - f_n \right\rVert_\infty + \left\lVert g - f_n' \right\rVert_1
  \end{align*}
  Quindi
  \[
    \left\lVert f^\ast - f_n \right\rVert_{\infty,BV} \le \left\lVert f - f_n
    \right\rVert_\infty + 2 \left\lVert g - f_n' \right\rVert_1
  \]
  Passando al limite per \(n \to +\infty\) si ha che \(f_n \to f^\ast\) in norma
  \(\left\lVert \cdot \right\rVert_{\infty,BV}\). In particolare, \(f^\ast = f\).

\item Abbiamo già visto che \(BV[0, 1]\) è uno spazio vettoriale: manca
  da dimostrare che questo spazio è chiuso anche per prodotto puntuale
  di funzioni e possiede un elemento neutro rispetto a questa
  operazione.

  Siano \(f, g \in BV[0, 1]\) e mostriamo che \(fg \in BV[0, 1]\). Fissiamo
  anche una qualsiasi partizione
  \(\mathcal{P} := \set{ 0 =: a_0 < \dots < a_N := 1 }\). Possiamo scrivere allora
  \begin{align*}
    V(fg, \mathcal{P}) &= \sum_{k=1}^{N} \left\lvert f(a_k)g(a_k) -
               f(a_{k-1})g(a_{k-1}) \right\rvert = \\
             &=  \sum_{k=1}^{N} \left\lvert f(a_k)(g(a_k) -g(a_{k-1})) +
               (f(a_k) - f(a_{k-1})) g(a_{k-1}) \right\rvert \leq \\
             &\leq \left\lVert f \right\rVert_\infty V(g, \mathcal{P}) + V(f, \mathcal{P}) \left\lVert g \right\rVert_\infty
  \end{align*}
  Quindi anche \(fg \in BV[0, 1]\). La funzione costante a \(1\) è
  ovviamente a variazione limitata ed è l'elemento neutro rispetto al prodotto.
  
\item Abbiamo visto che
  \(V(fg) \le \left\lVert f \right\rVert_\infty V(g) + \left\lVert g
  \right\rVert_\infty V(f)\).
  % Fissata una qualunque partizione
  % \(\mathcal{P} := \set{ 0 =: a_0 < \dots < a_N := 1 }\) di \([0, 1]\), si ha
  % \begin{align*}
  %   V(fg, \mathcal{P})
  %   &= \sum_{k=1}^{N} \left\lvert [f(a_k) -
  %     f(a_{k-1})]g(a_k) + f(a_{k-1})[g(a_k) - 
  %     g(a_{k-1})] \right\rvert \le \\
  %   &\le \sum_{k=1}^{N} \left\lvert f(a_k) -
  %     f(a_{k-1}) \right\rvert \left\lvert g(a_k)
  %     \right\rvert + \sum_{k=1}^{N} \left\lvert
  %     f(a_{k-1}) \right\rvert \left\lvert g(a_k) - 
  %     g(a_{k-1}) \right\rvert \le \\
  %   &\le \left\lVert g \right\rVert_\infty V(f) +
  %     \left\lVert f \right\rVert_\infty V(g)  
  % \end{align*}
  % Abbiamo quindi la disuguaglianza che ci serviva.
  Pertanto
  \begin{align*}
    \left\lVert f \right\rVert_{\infty,BV} \left\lVert g \right\rVert_{\infty,BV}
    &= \left\lVert f \right\rVert_\infty \left\lVert g
      \right\rVert_\infty + \left\lVert f \right\rVert_\infty V(g) + \left\lVert g
      \right\rVert_\infty V(f) + V(f)V(g) \ge \\
    &= \left\lVert fg \right\rVert_\infty + V(fg) = \left\lVert fg
      \right\rVert_{\infty,BV} .
  \end{align*}

\item
  \(\left\lVert f \right\rVert_{\alpha, BV} \le \left\lVert f \right\rVert_{\infty,
    BV}\). Viceversa,
  \(\frac{f(\alpha)}{\left\lVert f \right\rVert_\infty} \left\lVert f
  \right\rVert_{\infty, BV} \le \left\lVert f \right\rVert_{\alpha, BV}\).
\end{enumerate}



%%% Local Variables:
%%% mode: LaTeX
%%% TeX-engine: luatex
%%% ispell-local-dictionary: "italian"
%%% TeX-master: "main"
%%% End:

